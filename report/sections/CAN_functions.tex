%!TEX root = ../main.tex
\subsection{Functions of the GoCAN Protocol}\label{sec:CAN_functions}
Some additional functionality is required in GoCAN to fulfill the requirements set.
This functionality is described throughout this section.

\subsubsection*{Timestamps}
As mentioned in section~\ref{sec:CAN-bus} the CAN protocol is not real time, therefore all data on the bus must be timestamped before transmission. 
The resolution of the timestamps has been decided to be 1 ms, as this leaves sufficient accuracy for the sensors analysed in this project.
This presents several challenges, in terms of what the time reference is and how to convey timestamps with adequate precision without creating too much overhead.
With internet access, the Epoch timestamp is available. 
This is infeasible in bare-metal code.
Instead a counter will be implemented on each node, incrementing an 32 bit integer every 1 ms.
This gives 49.6 days before overflowing.\\

Because a standard 32 bit integer takes 4 bytes, it would introduce a large overhead if the timestamp is sent with every message. 
Instead, each timestamp will determine the time of all subsequent data messages, until a new timestamp is sent. 
As an example, the pseudo code below describes the transmission of all nine axis of the IMU with the same timestamp.
Additionally, the IMU can transmit pressure and temperature measurements, but as these are slower signals, they would be transmitted at a lower rate.

\begin{lstlisting}
Send Timestamp
Send Ax, Ay, Az
Send Gx, Gy, Gz
Send Mx, My, Mz
Wait
Send Timestamp
Send Ax, Ay, Az
Send Gx, Gy, Gz
Send Mx, My, Mz
Send Pressure, Temperature
Wait
\end{lstlisting}

Lines 2-4 are measurements taken at the timestamp in line 1.
The WiFi node appends the corresponding timestamps to each of these nine data points with the corresponding timestamp.
Lines 7-10 refer to the timestamp at line 6.

\subsubsection*{Commands}
The WiFi node is capable of issuing commands on the network.
As described in section~\ref{sub:CAN_protocol}, it is possible to start and stop any specific or all other nodes from the WiFi node.

\subsubsection*{Synchronization}
Another specific command is the synchronization command. 
When the system starts all nodes will start polling their Rx Fifo for this sync command. 
When this command is received, the node will start the millisecond timer.

\subsubsection*{Multiframe Messages}
Due to the limited data length of a CAN frame, it is likely necessary to support multiple frames per message. 
Generally it is better to use all 8 bytes of data in one frame, to reduce the relative size of the overhead, and in case a dataset isn't easily split into 8 byte portions, it might be easier to bundle it all together and send as multiple full frames. 
In the message ID of all but the first frame of a multi-frame message the \texttt{new message}, to make it highest priority.
That way a message will not be interrupted by other messages from other nodes.
The construction and interpretation of these multiframe messages are described in sections~\ref{sec:sensor_node} and~\ref{sec:frontend}.

\subsubsection*{Datatype and Scaling}
As most sensor data comes from sources of limited resolution, the optimal solution would be to send only the number of bits that are measured. 
In some cases however that is not possible, and a better solution is to round up or down to the nearest byte, and send the data as a fixed point data type.
Intead of defining a new datatype, an integer of appropriate length will be used instead, and the object dictionary will define the scaling. \\

As an example, the message described in table~\ref{tab:message17_OD} contains four data points of 2 bytes. 
This contains the current measurements on the three phases along with the electrical position of the rotor.
An example is described in table~\ref{tab:message17_OD}.

\begin{table}
	\centering
	\begin{tabular}{l|l|l|l|l|l}
		Message ID & DLC & $\mathrm{I_a}$ & $\mathrm{I_b}$ & $\mathrm{I_c}$ & $\mathrm{\Omega _e}$ \\ 
		\hline
		10100001001 & 1000 & 414 & -1545 & 1131 & 13090 \\
		\hline
		 & & 25.9 A & -96.6 A & 70.7 A & 1.31 rad
	\end{tabular}
	\caption{Example of message data from line 17 of table~\ref{tab:OD}}
	\label{tab:message17_OD}
\end{table}
According to table~\ref{tab:OD}, there are two scaling factors defined, one for currents and one for angle. 
The bottom line of table~\ref{tab:message17_OD} shows the resolved measurements.
