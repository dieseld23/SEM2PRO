%!TEX root = ../main.tex
\subsection{Functions of the CAN protocol}\label{sec:CAN_functions}
Some added functionality is necessary for the High Level protocol, which will increase the usability of the protocol.

\paragraph*{Timestamps}
As mentioned in section~\ref{sec:CAN-bus} the CAN protocol is not real time, which means all data on the bus must be timestamped. 
This way, the latency of each data point will be known, and it would be possible to work with data fusion.
The resolution of the timestamps has been decided to be 1 ms, as this leaves sufficient accuracy for the sensors analysed in this project.
This presents several challenges, in terms of what the time reference is and how to convey timestamps with adequate precision without creating too much overhead. 
For a node running Linux, the Epoch timestamp is available, but this is not available on baremetal code - it is also too coarse, as it only updates every second.
Instead a counter will be implemented on each node, incrementing an s32 integer every 1 ms.
This gives 24.8 days before overflowing.\\

Because a standard s32 integer takes 4 bytes, it would introduce a large overhead if the timestamp is sent with every message. 
Instead, each timestamp will determine the time of all subsequent data messages, until a new timestamp is sent. 
As an example, the pseudo code below describes the transmission of all nine axes of the IMU with the same timestamp.
Additionally, the IMU can transmit pressure and temperature measurements, but as these are slower signals, they would be transmitted at a lower rate.

\begin{lstlisting}
Send Timestamp
Send Ax, Ay, Az
Send Gx, Gy, Gz
Send Mx, My, Mz
Wait
Send Timestamp
Send Ax, Ay, Az
Send Gx, Gy, Gz
Send Mx, My, Mz
Send Pressure, Temperature
Wait
\end{lstlisting}

Lines 2-4 are measurements taken at the timestamp in line 1.
The WiFi node then saves each of these nine data points with the corresponding timestamp.
Lines 7-10 refer to the timestamp at line 6.\\
Because of the Tx Fifo in the CAN controller, it is possible to build and send four or five frames simultaneously, and let 


\paragraph*{Commands}

\paragraph*{Synchronization}

\paragraph*{Multiframe Messages}

\paragraph*{Datatype and Scaling}