%!TEX root = ../main.tex
\label{sub:implementation_of_sensors}
This section will describe the implementation of the sensors selected during the analysis.
%!TEX root = ../main.tex
%Motor data object: 4600h. includes motor slip frequency (not relevant), currents, voltages and temperature of heatsink. Don't know yet what the subindices are.
%It is possible to map them to Process Data Object for fixed time updates. It doesn't say anywhere what update rate, we can achieve.
%I've gotten a good amount of data from Karsten.

%This section assumes that CANopen has been adequately explained beforehand

\subsection{Interfacing with Sevcon}\label{sub:Sevcon_interfacing}
The Sevcon Gen4, currently on the go-kart, is compatible with CAN.
However, as it is a general purpose motor driver, it cannot be programmed to use GoCAN. 
For this reason, and to make the network unaffected by replacement of the motor driver, it has been decided to use a Zybo to interface with the Sevcon.

\subsubsection*{Physical Connection}\label{sub:sevcon_physical_connection}
Communication with the Sevcon is done through a high speed CAN bus that needs to adhere to ISO11898-2.
As described in section~\ref{sub:CANphys}, one transceiver board has two transceivers along with a terminal so that one Zybo can connect to the Sevcon using the second CAN controller.

\subsubsection*{Sevcon Object Dictionary}\label{sub:sevcon_object_dictionary}
The Sevcon utilizes CAN open, which means all of its parameters are listed in an object dictionary.
Because the Sevcon is a general purpose AC motor driver, its object dictionary is very large, and holds a lot of objects that are irrelevant for this particular setup, such as motor slip, and speed control parameters. 
The object directory is documented in a 1400+ line Excel file. 
Some objects of interest listed in table \ref{tab:parameters_of_interest}.

\begin{table}[h]
	\centering
	\begin{tabular}{| c | c |}
		\hline
		Parameters & Index-subindex \\ % Excel line
		\hline
		Motor Temperature & 4600h-3 \\ % 977
		Measured Id & 4600h-7  \\ %981
		Measured Iq & 4600h-8  \\ %982
		Measured Vd & 4600h-9  \\ %983
		Measured Vq & 4600h-10 \\ %984
		Target Id & 4600h-5    \\ %979
		Target Iq & 4600h-6    \\ %980
		Encoder Read-out & 4630h-9 to 12 \\ %1137
		Throttle value & 2620h \\ %330
		Velocity & 606Ch       \\ %1378
		\hline	
	\end{tabular}
	\caption{List of some of the parameters readable through CANopen}
	\label{tab:parameters_of_interest}
\end{table}

For the most part, these values have 16 bit resolution, which means they can be grouped together four at a time in a process data object.
The fact that a value can be mapped to a PDO means, that it can be transmitted to the Zybo at fixed time intervals or whenever it is updated.
The Encoder Read-out sin/cosine encoder position, so it needs to be converted to mechanical angle. 
This is done using equation~\ref{eq:cos_sin_to_degree}

\begin{equation}
\Omega_m = \mathrm{atan2}(\cos,\sin)
\label{eq:cos_sin_to_degree}
\end{equation}

These adaptations need to be done to make the Sevcon node a generic motor driver node.
That way it would be possible to use this system with a custom made inverter.

\martin{We need to write something about implementation of the other sensors. We also need to use the same order throughout the report}

\subsection{Interfacing with the IMU}\label{sec:interface_IMU}
The used IMU is a VectorNav vn-100 IMU.
The physical interface to the IMU is a usb cable and when connected to Linux it shows up in \texttt{/dev/}.
VectorNav provides an extensive C an C++ library for both Windows and Linux use \cite{vectornav}. 


\begin{lstlisting}[caption=IMU CODE......,label=code:rmc]
errorCode = vn100_getYawPitchRoll(&vn100, &ypr);
printf("%+#7.2f %+#7.2f %+#7.2f\n", ypr.yaw, ypr.pitch, ypr.roll);
\end{lstlisting}
\mikkel{Does it make any sense to include some example code from the internet?!}
\mikkel{More text here? We are not using the IMU for anything.}

\subsection{Interfacing with the GPS}\label{sec:interface_GPS}
\mikkel{Is it enough about this?}
The used GPS is a u-blox NEO-6P GPS module.
The physical interface to the GPS is a usb cable and the output adheres to the NMEA standard and is made of 8 different NMEA sentences.
The RMC sentence contains all essential information, that being position, velocity and time.
Therefore the implemented GPS class, that has the responsibility of interfacing the GPS, only needs to decode RMC sentences.
An example of a RMC sentence is shown in code snippet \ref{code:rmc}.
The extracted information that the time is 09:11:23:00, the module is active, latitude is 55 degrees 22.03929 minutes North, longitude is 10 degrees 25.91037 minutes East, the speed is 0.348 knots over ground, the date is 7. of November 2016 and checksum is 7C.
For the sake of simplicity all coordinates are converted to degress with decimals after being read. 

\begin{lstlisting}[caption=RMC sentence.,label=code:rmc]
$GPRMC,091123.00,A,5522.03929,N,01025.91037,E,0.348,,071116,,,A*7C
\end{lstlisting}
\mikkel{code line should be prettified !}

\subsubsection{Service virtualization}
The GPS only produces interesting data when it receives data from a number of satellites. 
This means that the GPS antenna needs to be outside, which is not very practical when developing software.
Therefore the output from the GPS with the antenna outside was piped into a file.
This file was then read by a program with a fixed time interval thus making a service virtualization of the GPS.
This service virtualization was used when developing software for the GPS.