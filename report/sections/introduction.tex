%!TEX root = ../main.tex
\section{Introduction} % (fold)
\label{sec:introduction}
As a link in the master's programme in electronics on University of Southern Denmark (SDU), an electric go-kart is made available to the students.
The go-kart serves as a development platform around which many of the projects throughout the programme are to be centered.
Long-term, lecturers have expressed a desire in potentially making the go-kart autonomous.
Creating autonomous vehichles is a vast subject involving several areas of science and engineering.
Many techniques such as computer vision and path finding need to be implemented in some form in order to provide the go-kart with sufficient information about its surroundings.
All of the algorithms used to make a vehicle autonomous share a common element.
They all, in some way, rely on data.
This data can be collected in a multitude of ways: cameras, LIDAR and encoders are just a few devices that may be used.\\
In order to work towards a goal as daunting as creating an autonomous vehichle from a platform as simple as the go-kart, clearly, it is necessary to divide the task into several, if not many, different projects.
%Depending on the method, implementation of an IMU and the appertaining filtering in itself is worthy of its own project.
This fragmentation of the tasks means that the work is likely to be done by a multitude of students spanning over several years.\\
This project aims to create a backbone for data collection for the go-kart system.
The idea of the system is to support communication among multiple nodes, which can be data producers or consumers, connected to a local network mounted on the go-kart.
The collected data is sent over wireless communication to a stationary computer, which then is presented to a user in a front-end.
The system has been designed to be modular at its core, making it possible for future students to extend it by adding more nodes and features, according to the needs of their projects.

\thomas{Generally, we agreed not to bring up the autonomous vehicles, and I am not against completely nuking this section, but I think it might be a descent way of introducing the subject of what we are doing.. Also, very abrupt ending, ran out of inspiration..}
% section introduction (end)
\catalin{
1.If we are to mention sth about the autonomous vehicle, maybe it should be described a little less than it is now.
2.Commented a line "Depending on the method, .. IMU... worthy of its own project.
3.}