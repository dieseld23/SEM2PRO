%!TEX root = ../main.tex
\section*{Preface}
\addcontentsline{toc}{section}{Preface}

\section*{Acknowledgment}
\addcontentsline{toc}{section}{Acknowledgment}
This project would not have been possible without the invaluable assistance and patience of our supervisor. 
A Tremendous thank you to Leon Bonde Larsen.
The process was simplified by Assistant Professor Kjeld Jensen from which the sensory equipment used in the report was supplied.
\vspace{5cm}
\begin{center}
	\begin{minipage}[t]{.55\textwidth}\large
		\begin{center}
		Catalin Ionut Ntemkas\\
		\vspace{1cm}
		\hrule
		\vspace{0.5cm}
		Martin Brøchner Andersen\\
		\vspace{1cm}
		\hrule
		\vspace{0.5cm}
		Mikkel Skaarup Jaedicke\\
		\vspace{1cm}
		\hrule
		\vspace{0.5cm}
		Thomas Søndergaard Christensen
		\vspace{1cm}
		\hrule
		\end{center} 
	\end{minipage}
\end{center}

\vspace{1.2cm}
  \begin{center}
    \textsl{The report, source code, schematics, data and plotting script can be found at:}  
    \end{center}
    \vspace{-5pt}
    \begin{center}
	\renewcommand{\UrlFont}{\color{black}\normalsize\tt}
    \url{https://github.com/catalin852/SEM2PRO_SDU}
   \end{center}
\newpage

\section*{Abstract}
\addcontentsline{toc}{section}{Abstract}
A go-kart has been supplied as a development platform for student projects at University of Southern Denmark.
In the interest of enabling more complex projects, a unified data collection system is necessary.
This is developed throughout this report. 
An analysis is done to determine the requirements of such a system.
It was found that a two-part network is suitable for this application.
The two parts are a CAN network and an ad-hoc WiFi network.
A custom protocol for use on CAN, GoCAN, is developed.
GoCAN supports up to 16 sensors from which data can be monitored on a remote monitoring station using the WiFi connection.
The system was only partially implemented since a connection between the CAN network and Linux was not achieved.