%!TEX root = ../main.tex

\section{Future Work}\label{sec:future_work}
Unfortunately this project did not amount to a finished working product, but given more time, it would be possible to make the CAN controller work on all nodes, independently of whether it runs Linux or bare-metal code.
In addition to this, there are some features which would be  highly beneficial to include.
All of this is discussed in this section.

\subsection*{Design CAN Controller on FPGA}
As the final implementation might be done in PL, it is possible to utilize the 8 Mb/s transfer rate of the CAN transceivers.
This does not conform with the ISO11898-2 standard, though, but a custom controller in FPGA can easily produce and interpret signals up to or beyond 8 Mb/s.
First objective of future work would be to implement the one missing link to make the entire project work.
This would be implemented on FPGA for two reasons;
the XCanPs controllers built into the Zybo can operate up to 1 Mb/s, but the transceivers can operate up to 8 Mb/s, so better to use the faster option.
Another reason is modularity.
An IP core could be included both on a bare-metal coded Zybo, and one running Linux.
So instead of working on two different methods for implementing a CAN controller in the PS, one method would work for all Zybos.

\subsection*{Startup Security}
As described in section~\ref{sec:CAN_functions}, the WiFi node sends out a sync signal when the whole network starts up.
The WiFi node will send this signal after a fixed time, but there is a risk, that not all nodes are ready to receive yet. 
An added security measure would be to add an "Ready for Sync" message, that all nodes will send out at a fixed interval.
The WiFi node will then need to know which to expect, and will not send synchronize the other nodes until all of them have reported ready.

\subsection*{Timestamping}
\mikkel{Brand new subsection!}
As described in section \ref{sec:time} the time stamping on the sensor nodes is done in userspace in Linux, which does not produce an  accurate timestamps.
The reason for using Linux on the nodes is mainly that USB and especially WiFi requires complex drivers that are already implemented in Linux.
As the frames are timestamped when they reach the WiFi node there is no requirement for it to be real-time.
This leaves a solution to the imprecise time stamping to be to let the WiFi node run Linux and the rest of the nodes to run on bare-metal code.
It of course requires a USB driver that can run on bare-metal code.

\subsection*{Extended Command Structure}
As described in section~\ref{sub:CAN_protocol}, the WiFi node can only transmit 4 different message types, because 4 bits are used to address the command. 
According to the OD in~\ref{tab:OD}, three of these are already in use; Sync, Start and Stop. 
This leaves only one more command to transmit.\\

A more prudent approach would be to restructure the commands, and possibly use the data field to convey more information. 
Four command IDs will be used: sync, command, set parameter, and get parameter. 
Sync will be as it is now, but command would be any commands to control specific nodes, such as start, stop or change mode. 
The data field would then contain one byte of data, determining which command is being sent.
The getting and setting of parameters, would require one or two bytes of the data field to determine which node is being manipulated.
Additionally, changing parameters on a go-kart while it is driving can be dangerous, so it will be important to include safety measures on the target node.

\subsection*{Internode Communication}
One of the advantages with both a bus, that all nodes can be snooping on all of the communication.
At this stage though, the nodes cannot interpret what the other nodes are sending out, but there might be situations where one node would want to get data from another.
One example could be velocity information from the front wheels being used by the motor controller to determine if the wheels are slipping, and reduce the torque accordingly. 
To reduce noise, it would be better to put a node closer to the front of the car, than to use long dedicated cables for each sensor in the front of the car, and then transmit the data over the CAN bus.\\

At this point, this is not possible, because one node has no knowledge of other nodes than itself and the WiFi node. 
This is to make the system more modular.
Although implementing internode communication would likely make the network more rigid, it would likely be possible to ensure that the front node in the above example could be changed, without having to make changes to the motor controller node.

\subsection*{Utilizing Asymmetric Multiprocessing}

A proposed solution for the problem of accessing the CAN controllers from Linux is implementing a mechanism called asymmetric multiprocessing.
It is supported by the Zynq-7000 AP SoC since there are two cores on the same processor which share common memory as well as peripherals.\\

The idea of this is to run Linux OS on one core, while on the other one a bare-metal system, which both of them can communicate with each other through shared memory. The core running Linux is needed to be set up as the master CPU since it is the one that starts the bare-metal system. Also, the operating system needs to be configured as symmetric multiprocessing with a maximum number of one CPUs.
It is a good approach because it will ensure that Linux configures the interrupt control distributor and the snoop control unit appropriately for multi-CPU environment, but only running on one of the two CPUs. Lastly, the shared memory should be handled properly to avoid conflicts.\\

It is a promising solution, especially suitable for multi-core embedded platforms where different systems need to run simultaneously. In the case of this report, it could have solved the communication difficulty between the CAN network and Linux, but due to lack of time, it was researched only at a theoretical level. For more details about the instructions on implementing this mechanism, the reader may refer to the Xilinx document \cite{Xilinx_AMP}.
