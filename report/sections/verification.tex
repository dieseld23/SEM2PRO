%!TEX root = ../main.tex

\thomas{Needs rewrite. Emphasis on service virtualization, also both describe the can and software side of the "gap}
Throughout this section tests will be designed and carried out in order to verify that the various subsystems live up to the requirements set in section \ref{sec:system_requirements}.\\

In order to show that the system can be fully functional despite the unsuccessful attempt to implement it, a different approach than the actual implementation is needed.
One way of doing that is to apply a method called service virtualization where one part of the system can be simulated to prove that the rest of the system is fully functional, given the presence of the missing simulated part.
\\
In this case, the system that was designed for this project can be divided into two parts or subsystems.
The first one is the physical CAN network to this point is only functional in bare\-metal code, which is described in section~\ref{sub:CAN_Bus_Tests}.
The second part would be considered the subsystem present on Linux which included the protocol designed to be used with the CAN bus and the application which extracted the data from the sensors.
This is described in section~\ref{sec:node_software}
The connection between the two subsystems is the service virtualization and could be replaced either by a software buffer.

%!TEX root = ../main.tex

\subsection{CAN Bus Tests}
With the CAN controller available on the PS of the Zybo, the CAN bus will be tested to determine the transmission abilities of the network.
The transceivers are able to work at up to 8 Mb/s, while the CAN controllers on the Zybo only guarantee functionality up to 1 Mb/s. 
As the final implementation should be done in PL, there is nothing preventing the controller working at 8 Mb/s -- the FPGA can easily produce and interpret signals up to or beyond 8 Mb/s.

\paragraph{Latency tests}
For this test, only two Zybos are needed.
One node prepares a frame for transmission, then sets a GPIO pin high.
The other node will then wait for the full message frame to be received, and then set its GPIO pin high.
Using an oscilloscope, it is possible to measure the time it takes to send a message.\\

The test needs to be run for the largest and smallest frames, that is one with 8 bytes of data, and one without data. 
The messages will be constructed, so that bit stuffing doesn't occur by writing 0x55 or 0xAA for each byte. 

\paragraph{Bandwidth}
This will be calculated, as the faster controllers are not available.
Bandwidth is considering the amount of net data being transmitted per unit time, when excluding the overhead.
Bandwidth will be calculated for 8 byte frames, and bit stuffing will be omitted.\\

The maximum operating data rate for the transceivers is 8 Mb/s.
As mentioned in section~\ref{sub:CanMessageFrame}, the CAN frame 47 bits of overhead. 
Including 8 bytes of data, this comes up to 111 bits. 
Time per frame is:

\begin{equation}
\frac{111}{8 \cdot 10^6} = 1.39 \cdot 10^-5
\end{equation}

As each frame contains 8 bytes of data, the data rate becomes:

\begin{equation}
\frac{8}{1.39 \cdot 10^-5}= 5.77 \cdot 10^5
\end{equation}

That means, that the effective transfer rate is 577 kB/s, or 4.61 Mb/s

\paragraph{Message filtering}
Three Zybos will be needed for this test.
One will act as a transmitter, and the two others will receive messages.\\
The transmitting node will shift back and forth between three message IDs, and write a recursive message, i.e. counting up from zero.
One receiving node will only accept one message ID, the other node will only accept another message ID, and the third message ID will be ignored by both.


\paragraph{Priority when multiple node sending}
Two Zybos will be needed for this test.
Both Zybos will be prompted to send a message when an interrupt occurs on an input pin. 
The data part of the messages don't matter, but the message IDs do; they need to be different.
Zybo A will send the message ID 0b10101000000.
Zybo B will have the message ID 0b10101010000.
That means, that Zybo A has higher priority, and Zybo B will stop trying to send a message when the seventh bit occurs, and receive instead.
It is assumed, that once Zybo A is done sending, (and its frame is acknowledged), Zybo B will succesfully retry its transmission. 
%!TEX root = ../main.tex
\subsection{Wifi}
%!TEX root = ../main.tex
\section{Node Software}
\label{sec:node_software}
When addressing the node software it is important to remember the distinction between sensor node software and wifi node software.
The software for the wifi node has not yet been fully implemented therefore this section will only address the verification of the functionalities of the sensor node software implemented on a GPS sensor node.
\mikkel{I will need to write this when I know more about how it should be written.}
\subsection{GPS Sensor Node}

%!TEX root = ../main.tex

\section{Frontend}
This section describes the test done to verify the functionality of the frontend.
A user should be able to monitor data created on the network.
Figure \ref{fig:frontendsetup} depicts an overview of the setup used in the verification.
An amount of GPS data is recorded and is presented to the GPS node using service virtualisation as shown in section \ref{sec:servicevirtualisation}.
A few of the recorded datapoints can be seen in figure \ref{fig:gpsdata}.
\thomas{Reference to note on service virtualisation} 
\thomas{section on why can connection failed}
The two parts, CAN-bus and WiFi node were not finished, see sections \ref{sec:somecansection} and \ref{sec:somewifinodesection} respectively, as such it was necessary to create service virtualisation for this link.
A small utility was written to serve this purpose.
This utility reads the messages sent by the GPS node, extracts the timestamp and inserts it into the data frame in place of the DLC nibble and then outputs it to stdio.
The message is then on the form seen in figure \ref{msg:backendmsg}, the form expected by the frontend.

\begin{figure}
	\missingfigure{overview of frontend setup}
	%\includegraphics[width=\linewidth]{graphics/frontendsetup}
	\caption{The setup used to verify the functionality of the frontend.}
	\label{fig:frontendsetup}
\end{figure}

In order to most accurately reproduce the actual function of the system, the test was done using a wireless connection between a Zybo and a PC.
The connection was established using the method described in section \ref{sec:wifi}.
On the PC, the frontend was started, taking its input from a socat listener
\begin{lstlisting}
>> ./frontend | socat - tcp-listen:2049
\end{lstlisting}
Then, on the Zybo, the GPS and 'WiFi' nodes were started, piping their output to socat:
\begin{lstlisting}
>> ./sensornode | ./verification | socat - tcp:2049
\end{lstlisting}

