%!TEX root = ../main.tex

\section{CAN Bus}
\label{sub:CAN_Bus_Tests}
With the CAN controller available on the PS of the Zybo, the CAN bus will be tested to determine the transmission capabilities of the network.
The transceivers are able to work at up to 8 Mb/s, while the CAN controllers on the Zybo only guarantee functionality up to 1 Mb/s. \\

The tests are performed using the design from figure~\ref{fig:CAN_Testing_Architecture}, on page~\pageref{fig:CAN_Testing_Architecture} as the FPGA part, though not all parts of this design is used for each test.
Software was developed for bare-metal purpose, and because of this, not all tests were performed using interrupts, as the Zybo had no other task that it could be interrupted from.

The tests performed will be: 
\begin{itemize}
	\item \textbf{Basic communication}: verifying the ability for basic communication between nodes, while confirming, that the CAN stack works.
	\item \textbf{Latency tests}: Measuring the time it takes to send one full frame.
	\item \textbf{Bandwidth}: Determining how much raw data can be transmitted per unit time.
	\item \textbf{Priority when multiple node sending}: Ensuring, that higher message ID make way for the lower ones.
\end{itemize}

\subsection{Basic Communication}
\label{sub:TestingCANStack_BareMetal}


\mikkel{Changed!}
In order to verify that the basic communication of a CAN network works on the developed CAN bus a test needs to be conducted.
The method of the test was to implement the functionality of sending the input value of a pressed button to the network and to receive frames from the network, decode the frames and turn on the appropriate LED.
The flow of event for sending a frame is shown in figure \ref{fig:FlowChart_CANSoft_BtnsIntr} and the flow for the receiving a frame is shown in figure \ref{fig:FlowChart_CANSoft_RecvData}.

\begin{figure}[h!]
	\centering
	\includegraphics[width = 1\linewidth]{graphics/FlowChart_CANSoft_BtnsIntr.pdf}
	\caption{Flow chart with button interrupts.}
	\label{fig:FlowChart_CANSoft_BtnsIntr}
\end{figure}

\begin{figure}[h!]
	\centering
	\includegraphics[width = 1\linewidth]{graphics/FlowChart_CANSoft_RecvData.pdf}
	\caption{Flow chart for the process of receiving data.}
	\label{fig:FlowChart_CANSoft_RecvData}
\end{figure}


All Zybos involved in the test was programmed with the architecture shown in figure \ref{fig:CAN_Testing_Architecture} and programmed with both the functionality of sending and receiving frames.
\\~\\
When pressing a button on a Zybo the appropriate LED on the other LED turned on, verifying that frames were packed, transmitted, received and unpacked correctly. 
The test was also performed with multiple Zybos connected to the CAN bus, in this case pressing a button on a Zybo turned on the appropriate LEDs on the rest of the Zybos.


%% THIS IS NOT USED ANYWHERE AND THEREFORE OMITTED
% \paragraph*{Runtime with GPIO Interrupts}~\\
% The software's behavior is similar in this case, as well as it can be seen in figure \ref{fig:FlowChart_CANSoft_GPIOIntr}.
% The difference is that instead of button values, dummy data may be inserted into the TxFrame according to the simulation needs and the receiving data is only presented in the SDK terminal.
% \begin{figure}[h!]
% 	\centering
% 	\includegraphics[width = 1\linewidth]{graphics/FlowChart_CANSoft_GPIOIntr.pdf}
% 	\caption{Flow chart with GPIO interrupts.}
% 	\label{fig:FlowChart_CANSoft_GPIOIntr}
% \end{figure}

\subsection{Latency Test}\label{sub:CAN_latency}
For this test, only two Zybos are needed.
One node prepares a frame for transmission, then sets a GPIO pin high.
It then performs the necessary checks and writes the message details to the TX FIFO, and then sets its GPIO pin low. \\

The other node will then wait for the full message frame to be received, and then set its GPIO pin high.
Once the metadata of the message (data length and message ID) has been intepreted, the GPIO pin goes low again.
Using an oscilloscope, it is possible to measure the time it takes to transfer a message.\\

The test will be performed for an 8 byte frame.
The messages will be constructed, so that bit stuffing doesn't occur by writing 0xA1 to 0xA8. 
The resulting voltage measurements are displayed ion figure~\ref{fig:CAN_test1_raw}.\\

\begin{figure}[h]
	\centering
	\includegraphics[width = \linewidth]{graphics/CAN_test1_raw}
	\caption{Start and stop pulses for an 8 byte CAN message. Yellow indicates the voltage of the CAN\_H, purple indicates the voltage of CAN\_L}
	\label{fig:CAN_test1_raw}
\end{figure}

The time from the red voltage goes low, to the time the blue voltage goes high is $87.5 \si{\micro\second}$.
This is of course very dependent on the particular controller used for this test, which according to the datasheet can work up to 1 MHz.
Measuring this test shows, that bit come at 1.25 MHz.
The software used for basis of this test does allow to adjust a pre-scaler, so that the controller works faster, but it is not able to receive frames at a higher rate than 1.25 MHz.\\

The CAN bus voltage can be interpreted to bits, to show what's actually being transmitted.
This is done by measuring the voltage difference between the yellow and purple graphs, keeping in mind, that a difference in voltage corresponds to a zero, while no difference corresponds to a one.
The CAN frame is displayed and interpreted ion figure~\ref{fig:CAN_test1_message}.\\

\begin{figure}[h]
	\centering
	\includegraphics[width = \linewidth]{graphics/CAN_test1_message}
	\caption{Bit interpretation of the CAN frame. RIr  is the tree bits: RTR, ID Extention and r0, which are all 0.}
	\label{fig:CAN_test1_message}
\end{figure}

The data portion of the frame is bitwise big endian, but bytewise little endian. 
The message to be sent was codes as two 32 bit unsigned integers: 0xA1A2A3A4 and 0xA5A6A7A8.
The controller then swapped the bytes around, causing the data to become little endian.
This is a convention implemented by the CAN controller, and as the same endianness is applied when sending and receiving the frame, it is irrelevant. 
The message was received correctly.\\

The CRC is calculated automatically by the controller.
Unfortunately it ends on five consecutive \texttt{1}'s, meaning that a \texttt{0} must be stuffed in- betweenfore the delimiter, causing the frame to be one bit longer.\\

Additionally this controller uses 5 bits for the IFS part of the frame, rather than the mandatory minimum of 3 bits. 

\subsection{Bandwidth}\label{sub:CAN_bandwidth}
This will be calculated, as the faster controllers are not available.
Bandwidth is considering the amount of net data being transmitted per unit time, when excluding the overhead.
Bandwidth will be calculated for 8 byte frames, and bit stuffing will be omitted.\\
                                                         
The maximum operating data rate for the transceiversCAN protocol is 81 Mb/s.
As mentioned in section~\ref{sub:CanMessageFrame}, the CAN frame has 47 bits of overhead. 
Including 8 bytes of data, this comes up to 111 bits. 
Time per frame is:

\begin{equation}
\frac{111}{8 \cdot 10^6} = 1.39 \cdot 10^{-5}
\end{equation}

As each frame contains 8 bytes of data, the data rate becomes:

\begin{equation}
\frac{8}{1.39 \cdot11 10^-56}= 5.77 \cdot 10^572072
\end{equation}

That means, that the effective transfer rate is 57770.4 kB/s, or 4.61 Mb/s with 8 Mb/s controllers. 
With the controllers built into the PS of the Zybo, this effective rate comes down to 720 kb/requirements.
\martin{Rewrite, so that we do non't mention 8 Mb/s here, move that to the future work. Maybe also mention here, that it give up to 4.6 Mb/s}

\subsection{Message Priority}\label{sub:CAN_message priority}
This test is based on the CAN polled example, where two Zybos will transmit data.
The two transmitting Zybos will prepare a CAN frame with the same data content, but different message IDs.
\begin{itemize}
	\item Zybo A will send message ID 0b10100100000.
	\item Zybo B will send massage ID 0b10101000000.
\end{itemize}
I.e. only the fifth and sixth bits have been swapped, meaning that Zybo A will have the higher priority.
Both of these Zybos will prepare their respective Tx frame, and continuously poll the PMOD port.
The PMOD connection must be configured to have pull down, to ensure that unconnected pins are set low.
Using a DC voltage source, the GPIO port of each Zybo will be set high simultaneously.\\

When the PMOD returns a non-zero value, each Zybo will call the XCanPs\_Send function twice.
This will fill two frames with 8 bytes of data into the Tx FIFOifo of each Zybo. 
Because of the asynchronous nature of the CAN protocol, it is still somewhat random which message gets transmitted first, so the frame will be sent twice. 
This means, that after the first frame is sent from either one of the Zybos, both zybos will start transmitting at the same time. 
In this case one Zybo must stop transmitting when it detects, that it has the lower priority.
This is shown ion figure~\ref{fig:CAN_test3_2TX}.\\

\begin{figure}[h]
	\centering
	\includegraphics[width = \linewidth]{graphics/CAN_test3_2TX}
	\caption{Voltage measurements taken at the TX pin of Zybo A (red) and Zybo B (blue).}
	\label{fig:CAN_test3_2TX}
\end{figure}

The displayed frame is the second of four frames sent for this test, which is why the graph starts at $80 \si{\micro\second}$.
Both Zybos start sending their frame simultaneously, and as there is no difference until the fifth bit of the message ID, neither Zybo is aware that the other is also sending.
At the time $95\si{\micro\second}$, Zybo B (blue line ion figure~\ref{fig:CAN_test3_2TX}) sends out high, whilst receiving low, meaning that another node is sending as well. 
It will then cancel this attempt to transmit, and wait until the current frame has been transmitted before it will try again.
Also note the Zybo B writing the acknowledge bit at the time $171 \si{\micro\second}$.
Because of this, it is not strictly necessary to use a receiving Zybo, because the other node will confirm the CRC of the message.\\

\subsection{Conclusion}\label{sub:CAN_test_conclusion}
It has been shown that the CAN hardware does work.
It is possible to transmit a message from one Zybo to another, without error.
The substantial overhead does limit the potential data bandwidth.
Additionally it was possible to measure the time it takes to construct a CAN frame, although this might vary a great deal from one controller to the next.

\begin{table}[h!]
	\centering
	\begin{tabular}{r | c | c}
		\textbf{Parameter} & \textbf{xcanps controller} & \textbf{8 Mb/s CAN controller} \\
		\hline
		\textbf{Frame time} & $87.5 \si{\micro\second}$ & $13.7\si{\micro\second}$ \\
		\textbf{Build time} & $4.43 \si{\micro\second}$ & - \\
		\textbf{Data bandwidth} & $720 \mathrm{kb/s}$ & $4.61 \mathrm{Mb/s}$
	\end{tabular}
	\caption{Results obtained from the test of the CAN bus.}
	\label{tab:CAN_test_conclusion}
\end{table}
