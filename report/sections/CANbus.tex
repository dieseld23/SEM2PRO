%!TEX root = ../main.tex

\section{CAN bus}\label{sec:CANbus}
The CANOpen protocol runs on the CAN bus.
It was originally developed in the 1980's by Bosch.
It is a multi-master network, where each node connects to a common bus, and any node is then able to broadcast data to all other nodes.
The bus offers 1 Mbit/s on a bus up to 40 m of length. \todo{Martin: is this with or without the substantial overhad?}
\subsection{Physical Layer}\label{sub:CANphys}
The physical layer has three main parts: The CAN controller, the CAN transceiver and the bus itself. \\

The CAN controller can be implemented as a standalone IC, or in many cases integrated in the node itself.
The Zybo supports CAN, and an IP core is available, so the CAN controller is already given.\\

The CAN transceiver is connected to the controller by RX and TX voltage signals, and connects to the bus through two differential ports. 
The transceiver must support the standard ISO11898-2, as this is what the Sevcon motor driver uses.
The device SN65HVD232 from Texas instruments support this standard, and is supplied with 3.3 V, so it can be plugged right into the Zybo, and still communicate with the Sevcon even though its CAN bus uses $\si{5 \volt}$.\\

The bus has to be made of twisted pair wires with a characteristic impedance of $\si{120 \ohm}$, and terminated at each end with a $\si{120 \ohm}$ resistor.
That means, that if the bus is broken at any point, no communication will work.
Alternately, it is possible to terminate each node, but this greatly reduces transmission speed.