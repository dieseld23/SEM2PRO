%!TEX root = ../main.tex

\subsection{Power Requirements}\label{sec:power_requirements}
The Zybos need to be supplied with 5 V, which subsequently powers the sensors, the CAN busses and the WiFi.
The Sevcon has a 5V output, but it can only supply $\si{100 \milli \ampere}$, which is nowhere near enough.
It is therefore necessary to use a DC-DC converter to supply the nodes.
Some tests along with datasheet lookups has been to determine the current needed for this system.\\

By measurement, it has been found, that one Zybo board draws a maximum of $\si{475 \milli \ampere}$ while running linux.
The CAN transceivers consume up to 40 mA\cite{3.3V_CAN}. 
The GPS module consumes up to 40 mA, the IMU consumes up to 56 mA. 
Generally it is assumed that the sensors do not consume more than 75 mA per node, which brings the total current consumption up to 600 mA.
For 16 nodes, this brings the total current up to 9.6 A.\\

The source for this DC-DC converter will be the batteries on the Go-kart. 
These vary depending on charge from 57.6 down to 40 V, so the DC-DC converter would need to be operable in this entire range.
One example could be Mean Well SD-50C-5, which lives up to these requirements.

\begin{table}[H]
	\centering
	\begin{tabular}{r|l|l}
		Parameter & Requirement & SD-50C-5 \\
		\hline
		Input voltage & 40-57.6 & 36-72 \\
		Output voltage & 5 & 5 \\
		Output Current & 9.6 & 10
	\end{tabular}
\end{table}