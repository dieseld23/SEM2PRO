%!TEX root = ../main.tex
\catalin{IMPORTANT: In case we get the actual network running, don't forget to change the text accordingly (implementation not successful etc.)}
\subsection{Implementing CAN on Linux}

The next step after testing the stack board and showing the basic functionality of the CAN network was to implement it using Linux running on the Zybo board.
This step required enabling the CAN drivers that reside in the Xillinux kernel and gaining access to the CAN devices.
This section describes the necessary procedures that were followed.
At this point the reader should be informed that although the CAN controllers could be used on the Processing System using bare-metal code, the implementation of CAN on the Programmable Logic and the access to it from Linux was not successful.
After a lot of effort to create a physically functional network and researching on how to implement it, the conclusion that the available documentation is lacking was reached. 
Alternatively, other means to prove the concept of the project were taken into account.

\subsubsection{Enabling the CAN Drivers}

In order to enable the drivers on the Zynq7 Processing System, the Linux CAN driver guide \cite{Xilinx_wiki_Linux_CAN_driver} on the Xilinx wiki website \cite{Xilinx_wiki} was followed.
The Kconfig file under the path \ref{code:can_kconfig_pathfile} needed to be configured.
The entry at line 128 was changed as seen in the code snippet \ref{code:can_kconfig_contents_line128}.
Originally, lines 130 and 131 were as seen in the snippet \ref{code:can_kconfig_original_line130}.

\begin{lstlisting}[caption={CAN Kconfig pathfile.},numbers=none,label=code:can_kconfig_pathfile]
/usr/src/kernels/3.12.0-xillinux-1.3/drivers/net/can
\end{lstlisting}

\catalin{Xilinx CAN has "" around it, but LaTeX shows errors if inserted in the code snippet.}
\catalin{Why do I get some questionmarks under references?"}

\begin{lstlisting}[firstnumber=128,caption={Kconfig file contents from line 128.},label=code:can_kconfig_contents128]
config CAN_XILINXCAN
	tristate Xilinx CAN
	depends on NET [=y] && CAN_DEV [=y] && CAN [=y] && (ARCH_ZYNQ || MICROBLAZE [=y])
	default y
	---help---
	  Xilinx CAN driver. This driver supports both soft AXI CAN IP and
	  Zynq CANPS IP.
\end{lstlisting}

\begin{lstlisting}[firstnumber=130,caption={Original content of lines 130 and 131.},label=code:can_kconfig_original_line130]
	depends on CAN && (ARCH_ZYNQ || MICROBLAZE)
	default n
\end{lstlisting}

Next, the modification of the device tree settings file was the second step, requiring an entry for the CAN PS to be inserted. The necessary file was located under the boot folder named as seen in \ref{code:dts_file_zybo}.

\begin{lstlisting}[numbers=none,caption={Device tree settings file and its path.},label=code:dts_file_zybo]
/boot/xillinux-1.3-zybo.dts
\end{lstlisting}

\begin{lstlisting}[caption={CAN Kconfig pathfile.},numbers=none,label=code:can_kconfig_pathfile]
/usr/src/kernels/3.12.0-xillinux-1.3/drivers/net/can/
\end{lstlisting}
\mikkel{Maybe you can provide some thoughts about the problem. Maybe also some words about the patching Xilinx-Digilent etc.}
As was previously mentioned, the implementation was unsuccessful. At the time of writing this report, 

\subsubsection{Utilizing the AXI CAN core}
\catalin{Should I show the arch design from Vivado with the AXI CAN?}
Another way to implement a CAN network was to use the AXI CAN core that is available in the Vivado Suite, instead of gaining direct access to the CAN drivers. Unfortunately due to a license restriction from Xilinx, the core can only be used for simulation purposes and not actual hardware implementations.

\catalin{Will opencores.org provide me with the valuable access we are currently seeking for this divine project? If yes, make a subsubsection}

\subsubsection{Using can-utils for virtual nodes}

During the development of this project, the can-utils tool was also used. It is a testing tool that can be executed on Linux and can be applied on real as well as virtual CAN devices. Initially it was used to create a virtual network with two nodes locally on a computer to gain a better understanding of how CAN networks function and how they can be configured. It was also intended to be used to test the actual network, but since the network was not implemented, further use of the tool was not necessary.
\catalin{Rerun some virtual tests with can-utils to get code snippets}
