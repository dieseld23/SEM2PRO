%!TEX root = ../main.tex
Throughout this report has been described the creation of a system for gathering and monitoring data from the SDU go-kart.
Here will be presented a consollidation of the outcome of the different parts of the project, as well as some of the observations with respect to the working methods employed by the authors.
\\~\\
Initially, an attempt was made to exemplify the requirements of a system such as the one being developed.
Each part of the system was analysed to this effect.
While the analysis does touch on every part of the project, it was not until late in the project that the value of the analysis was clear to the authors.
Some mistakes were made to this effect. 
For instance, on making the connection between the PC and the Zybo, three different methods were tried before finally settling on socat.
A process that might have been avoided with proper analysis of the problem at hand.
Even with the detours, a collection of requirements were set.
Many of which were verified in part \ref{part:verification}.

\section{State of the Requirements}
The results of the verification is consolidated here to provide an overview of the state of the project.
Each requirement is listed with an explanation as to whether the requirement was met.
Requirements marked with \cmark were fully realised, -- partially and \xmark ~are not realised.

\subsection{Functional Requirements}
\begin{itemize}
	\item[\cmark] Read data from sensors/data producers:
	Data was successfully read from a GPS module.
	The data was gathered using the real GPS module but service virtualisation was utilised in verifying the functionality of the node responsible of handling the GPS.
	This is explained in section \ref{sec:node_software}.
	\item[\cmark] Timestamp all data:
	A system was devised such that a packet contain the tinestamp of the data is sent prior to sending the data.
	The timestamp was left out of the actual data packet to limit the bandwidth spent on transmitting timestamps.
	This is explained in section \ref{sec:node_software}.
	\item[\cmark] Transmit data wirelessly between Zybo and monitoring station:
	This was succesfully done using a WiFi connection.
	An ad-hoc network was created using a WiFi/USB dongle, which is connected to from a PC.
	Verification of this feature can be seen in section \ref{sec:wifiverification}.
	\item[\xmark] Log data to SD card:
	Due to time constraints, it was not possible to implement this feature.
	\item[\xmark] Transfer logs to monitoring station:
	Since a log of the data is not kept, this requirement could not be fulfilled.
	\item[\cmark] Start/stop transmission from nodes:
	This feature is part of GoCAN, the protocol developed in the report.
	The user can issue a command which is relayed from the monitoring station to the CAN network using the WiFi node.
	This is shown in section \ref{sec:node_software}.
	\item[\cmark] Present data to user:
	A basic program reading the GPS data was written to act as a placeholder GUI.
	The output of this program is shown in section \ref{sec:frontendverification}.
	\item[\cmark] Provide an API for data access:
	A class was written to provide an API which acts as a front end to the underlying system.
	A function was written to interpret the GPS data and present it in a human readable format.
	This API was utilised in verifiying the previous requirement.
	\item[--] Must continue to function on node failure:
	This requirement is fulfilled only partially. 
	The network is agnostic to the number of nodes present at any one time and the failure of any sensor node will not bring the system down.
	The WiFi node does, however, provide the link between the CAN network and the monitoring station.
	Due to this, if the WiFi node fails, the system can no longer be used in monitoring data.
\end{itemize}

\subsection{Operational Requirements}
\begin{itemize}
	\item[\cmark] A developer can add new nodes:
	A framework has been created which allows the developer to add the necessary code to support nodes.
	Guidelines on how to implement new nodes are provided in appendix \ref{app:addnode}.
	\item[\cmark] A developer can modify existing nodes:
	Since the code for the framework mentioned in the previous requirement is available to the developer, it will also be possible to modify the code for any existing node.
	\item[\cmark] A developer can add API for specific sensor:
	The framework provides a front end class in which functions for accessing data from a particular node can be written.
	An outline of the process of adding to the API can be seen in appendix \ref{app:addnode}.
	\item[\cmark] A developer can add custom (G)UI:
	S
	\item A user can request a data log:
	\item A user can start/stop data transmission from nodes:
\end{itemize}

\subsection{Quality of Service Requirements}
\begin{itemize}
	\item Can support up to 16 nodes:
	\item Has wireless range greater than 80m:
	\item CAN network has at least 1Mb/s data bandwidth:
	\item[\xmark] Timestamps with a precision of 1 ms:
	While timestamping is done, it is currently not possible to guarantee 1 ms precision.
	This is due to the node being run on an ordinary Linux system.
	If the data gathered on the network is to be useful in any data processing application, it is necessary to port the system to a real time platform.
\end{itemize}

\subsection{Design Requirements}
\begin{itemize}
	\item Must allow for integration of new nodes:
	\item Software must be modular to allow for simple integration of nodes:
\end{itemize}

