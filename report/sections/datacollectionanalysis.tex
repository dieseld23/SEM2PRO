%!TEX root = ../main.tex

\section{Data Collection}
\label{sec:data_collection}
As described in previous sections, the system will have to support the use of several data producers.
In addition to this, it should also be possible to easily add additional data producers at a later time.
It is possible to implement the support of many producers on a single platform, however, eventually the platform would likely run out of I/O for additional support.
The single-platform approach would also likely not meet the computational requirements set by the increasing amount of equipment.
This implies that a multi-platform system is the appropriate approach for this system.
Each platform, or node, would then implement the functionality of one or more data producers.\\

The data needs to be collected from all of the nodes, then logged and transmitted wirelessly to the monitoring station.
The type of network depends greatly on the number of nodes that it has to support.
An ongoing project on SDU, Formula Student \cite{formulastudent}, has a similar platform for which a network has been developed.
That project, which is to develop a race car, has a network of seven physical nodes, distributed around the car, which each handle data producers and consumers in that area.
The physical distribution of the nodes means that the responsibility of a given group of sensors can be given to a node based on their placement on the car.
The parameters found in section \ref{sec:parameters} can be measured using five different types of sensors.
This project does, however, aim to give users the ability to add their own nodes.
For this reason the limit of supported nodes is set to 16.
It is worth noting that the structure of each node is irrelevant, so long as it adheres to the networks communication standards.
This means that a node can potentially comprise several (different) sensors or perhaps a complete separate network of nodes.

\subsection{Network Topology}

There are various network topologies that can be used to setup the required node network for this project.
These include among others the bus, ring and star. 
Before choosing a topology, a brief description of the purpose and functionality of the network as well as an overview of their advantages and disadvantages are needed.\\ 

The purpose of this network is to accommodate multiple nodes, such as sensors, sub-networks and in general data-producers.
These nodes need to be able to transfer their data and receive commands, as specified in the use cases.
\\
The communication between the various nodes does not require a central hub.
Furthermore, in the case that one node fails, the network as a whole should still be operative.
Since it is a multi-node network and it may require more nodes in the future, scalability is also required.

\subsubsection*{Network Topologies}
Networks can be described by different patterns, or topologies, some of which might have several technologies.
Some of these topologies, that are practical to implement, are listed bellow.
\begin{itemize}
\item \textbf{Bus:}
On this network, all nodes communicate through a common bus. 
This means that one node communicates to all other nodes at the same time, and that only one node can transmit at a time.
\item \textbf{Ring:} All the nodes in this type of network form a ring, where each node is connected to its two neighbours.
Messages travel along the ring.
\item \textbf{Star:} The star topology is a centralized type of networking.
All nodes are connected to a central hub, handling all the communication between them.
It offers high bandwidth and is easy to implement.
The major disadvantage is that if the hub fails, the entire network fails.
It can be scalable to a certain point, since the hub can be upgraded to handle more connections.
\end{itemize}

\subsubsection*{Suitable Topology}
%\martin{Like i todo'ed above, we should only concern ourselves with bus, ring and star, and at this point we should come to the conclusion that star does not work, and that ring and bus is really a toss up.}
\martin{rewrite}
For the system's networking purpose, the mesh type is not suitable, since it adds extra hardware requirements.
A node may be a simple sensor with a small micro controller and hence, connecting it to such a network is not feasible.
%In our system, a small embedded board computer will be the ov-computer which requires to maintain its connection with the network even in case of failures and also possibly the central hub in centralized topologies that such as the star and the tree.
The star network requires a lot of connections which normally are not present at standard micro controllers.
%Thus, these two topologies are not suited, since in case of its failure, the whole network fails.
Thus, the mesh and star networks are not suited for this system.
Scalability is also a requirement for the future connection of nodes.
Although the majority of the types provide a level of scalability, the addition of extra nodes always decrease the overall performance of any network.
Hence, a topology that balances the decrease in performance against the network's expansion is best suited for the project.\\
\martin{bus and ring fits best, subnetworks are irrelevant}
The approach that fits the requirements is implementing a bus network, where each node may be a subnetwork of a different type depending on the needs, making it into a hybrid network having a bus topology as its basis.
This type provides a good balance of reliability, scalability, hardware requirements and communication speed in comparison to the others.
\subsection{Networking technology}
In the previous section it was found that the bus and ring topologies in particular lend themselves well to the implementation of this system.
A number of different technologies exist, that implement these topologies. 
This section explores a few of the ones that are considered for this project to find the one best suited.

\subsubsection*{Selecting Technologies}
Different networking technologies exist in use today, such as CAN and Powerlink among others.
Powerlink is the network topology used by Formula Student; it is a ring type network, using Ethernet between neighbouring nodes.
Messages are transmitted as a pulse train from one node to its neighbour, who then appends its data to the end of this train. 
Message trains are sent out at fixed time intervals.
However, message trains need to be fully received by one node before it can be transmitted with its data appended, and because of this it can be an issue synchronizing the nodes.
There can be very little overhead, and with the Ethernet cables and ports setting the limits, the bandwidth is very large -- 100 MB/s on Formula Student.
However, this does require all nodes to have two Ethernet ports with direct access to an FPGA, and that is rare to find on an evaluation board.\\

CAN is a bus topology widely used in the automotive industry with data rates up to 1Mbit/s for small network lengths.
This is slightly less than the requirement calculated in section~\ref{sec:data_rate}, however that number was based on the highest possible sample rate of the IMU, along with the inefficient data type, which is 32 bit float.
Because CAN is a bus topology, it is easy to add another node to the bus, as one would just tap into the bus anywhere.
It is because of this tapping into the bus, that it is not suitable for higher data rates, and if the bus breaks at any one point, communication is lost on the entire bus.
Unlike Powerlink, this runs without a dedicated master.
It is not nearly as fast as Powerlink, and does not have the redundancy of a ring network.
Due to the use of buffers, the CAN bus is not a real time bus, but synchronization is easier.\\

The open source project openPowerlink offers FPGA implementation of Powerlink, and external modules can be utilized to handle the hardware part of the network.
This will allow evaluation boards with less than two Ethernet ports to work in the ring network. 
CAN also requires one external piece of hardware, but does not necessarily require an FPGA to run. 
According to the data rate calculations in section~\ref{sec:data_rate}, the CAN network has enough bandwidth to handle the requirements of this project. 
Because it is easier to implement CAN, and to add nodes to it, it has been decided to use a CAN network.
\martin{easer because less hardware, cheaper, powerlink is propriatary}