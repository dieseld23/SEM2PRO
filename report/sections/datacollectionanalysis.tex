%!TEX root = ../main.tex

\section{Data Collection}
\label{sec:data_collection}
As described in previous sections, the system will have to support the use of several data producers.
In addition to this, it should also be possible to easily add additional data producers at a later time.
It is possible to implement the support of many producers on a single platform, however, eventually the platform would likely run out of I/O for additional support.
The single-platform approach would also likely not meet the requirements of the computational requirements set by the increasing amount of equipment.
\martin{requirements of the requirements, yo dawg}
This implies that a multi-platform system is the correct approach for this system.
Each platform, or node, would then implement the functionality of one or more data producers.\\
The data needs to be collected from all of the nodes, then logged and transmitted wirelessly to the monitoring station.
The type of network depends greatly on the number of nodes that it has to support.
A project on SDU, Formula Student, has a similar platform for which a network has been developed.
\martin{One of the advantages of multi platform (and one that FS uses) is the ability to put the node close to the sensors/acutators}
On that platform seven nodes are connected on the network.
The parameters found in section \ref{sec:parameters} can be measured using five different types of sensors.
This project does, however, aim to give users the ability to add their own nodes.
For this reason the limit of supported nodes is set to 16.
It is worth noting that the structure of each node is irrelevant, so long as it adheres to the networks communication standards.
This means that a node can potentially comprise several (different) sensors or perhaps a complete separate network of nodes.

\subsection{Network Topology}

There are various network topologies that can be used to setup the required node network for this project.
These include the ring, bus, mesh, star and tree network topology. 
Before choosing a topology, a brief description of the purpose and functionality of the network as well as an overview of their advantages and disadvantages are needed. 
\mikkel{Maybe it should be boiled down}b
\subsubsection{Purpose of the Network}
The purpose of this network is to accommodate multiple nodes, such as sensors, sub-networks and in general data-producers.
These nodes need to be able to transfer their data and receive commands from the master node.
\martin{at this point, has it been established what the WIFI node is? I can't find it if it has. Ok, now I changed it to master node, and will determine the master node to be the wifi node in section~\ref{sec:EP}.}
The reasons for this is, that the use cases specify that sensor data should be transferred wirelessly and it needs to be possibly to start and stop specific nodes.
\\
\martin{That line bothers me. It's kinda explaining the reason to include something with use case, which should be inferred}
The communication between the various nodes and the wifi node does not require a central hub.
Furthermore, in the case that one node fails, the network as a whole should still be operative.
Since it is a multi-node network and it may require more nodes in the future, scalability is also required.

\subsubsection{Different Topologies}~\\
\todo[inline]{Thomas: This section needs to be cleaned of any statements such as "this is the simplest and cheapest...". They are broad conclusions that we have no merrits on saying}
\catalin{I understand for "cheapest", but simplest stands. Bus topology is the simplest. It is networks theory.}
\martin{I agree that "simplest" is not necessarily true. Obviously the easies would be star, as we can add a network switch. So I changed the part about bus and ring.}
\begin{itemize}
\item \textbf{Bus:}
On this network, all nodes communicate through a common bus. 
This means that one node communicate to all other nodes at the same time, and that only one node can transmit at the same time.
Because of this, it's easy to add another node to the bus, as one would just tap into the bus anywhere.
However, this bus with multiple taps sets limitations on the data rate, especially over greater distances of several tens of meters.
Also, the whole bus might break down if the connection is severed anywhere, even if there is still a wired connection between two nodes.
\item \textbf{Ring:} All the nodes in this type of network form a ring, where each node is connected to its two neighbours.
One node sends its data to its neighbour. The neighbour then appends its data, and sends data from both nodes on through the network. 
Messages travel along the ring, and eventually ending up at the wifi node.
This gives added security -- if one node or one connection breaks, the messages can travel the other way. 
Compared to the bus, a ring network can usually handle data rates orders of magnitude higher than the bus by communicating through ethernet cables. 
It is scaleable in the same way as the bus, as one can just insert another node between two nodes.
\item \textbf{Mesh:} In this type, each node has direct connections to each other node present in the network.
That provides speed and reliability in case of node failures, but requires more hardware and processing power to manage the connections.
It is very robust but scalability is certainly not one of its features, as the number of connections escalate by the number of nodes.
\item \textbf{Star:} The star topology is a centralized type of networking.
All nodes are connected to a central hub, handling all the communication between them.
It is fast, easy to implement and offers great reliability in case of node failures.
\martin{How does this offer reliability? IF connection to a node is severed, the node is gone.}
The major disadvantage is that if the hub fails, the entire network fails.
It can be scalable to a certain point, since the hub can be upgraded to handle more connections.
\item \textbf{Tree:} The last type, the tree is an extension of the bus and the star topologies combined.
It can be easily scalable, but that adds extra difficulty in its maintenance.
It also requires a lot of connections and in case the top (root) node fails, then all the network is down as well.
\item \textbf{Hybrid:} The last type is the hybrid one, where depending on the needs and purpose of the network, two or more topologies can be combined to achieve the best balance between their advantages and disadvantages.
\end{itemize}
\martin{This list should really only be Bus, Ring and Star, because the others aren't really realistic to implement}

\subsubsection{Suitable Topology}
\martin{Like i todo'ed above, we should only concern ourselves with bus, ring and star, and at this point we should come to the conclusion that star does not work, and that ring and bus is really a toss up.}
For the system's networking purpose, the mesh type is not suitable, since it adds extra hardware requirements.
A node may be a simple sensor with a small micro-controller and hence, connecting it to such a network is not feasible.
%In our system, a small embedded board computer will be the ov-computer which requires to maintain its connection with the network even in case of failures and also possibly the central hub in centralized topologies that such as the star and the tree.
The star network requires a lot of connections which normally are not present at standard micro-controllers.
%Thus, these two topologies are not suited, since in case of its failure, the whole network fails.
Thus, the mesh and star networks are not suited for this system.
Scalability is also a requirement for the future connection of nodes.
Although the majority of the types provide a level of scalability, the addition of extra nodes always decrease the overall performance of any network.
Hence, a topology that balances the decrease in performance against the network's expansion is best suited for the project.\\

The approach that fits the requirements is implementing a bus network, where each node may be a subnetwork of a different type depending on the needs, making it into a hybrid network having a bus topology as its basis.
This type provides a good balance of reliability, scalability, hardware requirements and communication speed in comparison to the others.
\subsection{Networking technology}
After discussing the different topologies, it is necessary to look at what technologies are readily available, and then decide which one fits the needs best.
\thomas{I think we need to discuss the topology/technology sections. In my view, it matters very little what topologies we like - what matters is what technologies are available, if that is of ring/bus/whatever topology then fine, we can discuss that.}
\thomas{We have 2 pages discussing the abstract of topologies but only 1/4 page discussing pros and cons of the different actual technologies?}
\subsubsection{Selecting Technologies}
Different networking technologies exist in use today, such as Ethernet, among others CAN and Powerlink.\\
Powerlink is the network topology used by Formula Student; it is a ring type network, using ethernet between neighbouring nodes.
Messages are transmitted as a pulse train from one node to its neighbour, who then appends its data to the end of this train. 
Message trains are sent out at fixed time intervals.
However, message trains need to be fully recieved by one node before it can be transmitted with data appended, and because of this it can be an issue synchronizing the nodes.
There can be very little overhead, and with the ethernet cables and ports setting the limits, the bandwidth is very large -- 100 MB/s on Formula Student.
However, this does require all nodes to have two network ports with direct access to an FPGA, and that is rare to find on a evaluation board.\\

CAN is widely used bus topology in the automotive industry with data rates up to 1Mbit/s for small networks lengths.
Unlike Powerlink, this runs without a dedicated master, meaning that there is one less thing that can go wrong.
Compared to Powerlink, it is not nearly as fast though, and does not have the redundancy of a ring network.
Due to the use of buffers, the CAN bus is not a real time bus, but synchronization is easier.\\

\martin{I adapted this. But one thing though, openPowerlink is not implemented on FPGA, which powerlink it self really should be.}
Powerlink can be implemented using openPowerlink at a software level and utilizing an external module to handle the hardware part of the network.
This will allow evaluation boards with less than two ethernet ports, can work in the ring network. 
CAN also requires one external piece of hardware, but does not necessarily require an FPGA to run. 
According to the data rate calculations in section~\ref{sec:data_rate}, the CAN network has bandwidth enough to handle the requirements of this project. 
Because it is easier to implement CAN, and to add nodes to it, it has been decided to use a CAN network.