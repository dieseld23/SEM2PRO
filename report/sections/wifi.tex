%!TEX root = ../main.tex
The off-kart network is done using wifi.
It was chosen to use this method as all modern laptops are equipped with the capability of connecting to such a network.
Hereafter the challenge lies in establishing an ad-hoc network between a zybo board and a PC.
This network is created using a USB-Wifi dongle on the Zybo board. 
The following script is run to bring down the device, change the network type to ad-hoc, bring up the device, create the ad-hoc network and, finally, assign an IP-address to the device.
Note: some commands require super user access.
For clarity, \texttt{sudo} is left out from all calls throughout this section.
\begin{lstlisting}[language=bash]
>>service network-manager stop
>>ip link set wlan0 down
>>iw wlan0 set type ibss
>>ip link set wlan0 up
>>iw wlan0 ibss join go-kart 2412
>>ip addr add wlan0 196.178.10.10
\end{lstlisting}
It is verified working using the ping command from both the Zybo board and to the PC and vice versa.

Upon setting up the ad-hoc network, a socket connection is created between the Zybo board and the PC using socat
\begin{lstlisting}[language=bash]
zybo>>socat - tcp-listen:2048
pc>>socat - tcp:196.178.10.10
\end{lstlisting}

Messages can be passed through this connection by piping data to it.

Following is a description of the steps taken to reach the approach explained above.
\subsection{Setting Up the Ad-Hoc Network}~\\
Since the Zybo is already running a Linux distribution which provides all of the necessary drivers, it was decided to use a WiFi dongle to create the ad-hoc network on the Zybo.
The highlights of the consequences of this choice are discussed below.
\\~\\
The first step was to ensure that the correct drivers are present on the system.
The USB/WiFi dongle is a TP-LINK TL-WN722N.
This device uses the Atheros AR9271 chipset and is on the list of supported devices for the ath9k\_htc driver (available on Xillinux) on \cite{ath9k}.
It is also capable of supporting ad-hoc networks, a crucial feature in setting up this network.
Running \texttt{dmesg} prints the kernel log revealing, amongst other things, what drivers are loaded.
\begin{lstlisting}[language=bash]
>> dmesg | grep ath
[10.329564] usb 1-1: ath9k_htc: Firmware htc_9271.fw requested
[10.332438] usbcore: registered new interface driver ath9k_htc
\end{lstlisting}
Additionally:
\begin{lstlisting}[language=bash]
>> lsusb | grep Ath
Bus 001 Device 002: ID 0cf3:9271 Atheros Communications, 
	Inc. AR9271 802.11n
\end{lstlisting}

As per these commands the operating system (OS) has correctly detected and loaded the driver.
The \texttt{iproute2} package contains the utilities used to manipulate TCP/IP connections.
\begin{lstlisting}[language=bash]
>> ip link show
1: lo: <LOOPBACK,UP,LOWER_UP> mtu 65536 qdisc noqueue 
	state UNKNOWN mode DEFAULT group default 
   link/loopback 00:00:00:00:00:00 brd 00:00:00:00:00:00
2: eth0: <BROADCAST,MULTICAST,UP,LOWER_UP> mtu 1500 qdisc 
	pfifo_fast state UP mode DEFAULT group default qlen 1000
   link/ether 00:0a:35:00:01:22 brd ff:ff:ff:ff:ff:ff
3: wlan0: <BROADCAST,MULTICAST> mtu 1500 qdisc mq 
	state DOWN mode DEFAULT group default qlen 1000
   link/ether ec:08:6b:1b:41:b3 brd ff:ff:ff:ff:ff:ff
\end{lstlisting}

The device has been given the identifier \texttt{wlan0}.
It can be brought up by:
\begin{lstlisting}[language=bash]
>> ip link set wlan0 up
>> ip link show wlan0
3: wlan0: <NO-CARRIER,BROADCAST,MULTICAST,UP> mtu 1500 qdisc 
	mq state DOWN mode DEFAULT group default qlen 1000
   link/ether ec:08:6b:1b:41:b3 brd ff:ff:ff:ff:ff:ff
\end{lstlisting}
At this point the device is up, as can be seen on the list of flags:
\begin{lstlisting}[language=bash]
<NO-CARRIER,BROADCAST,MULTICAST,UP>
\end{lstlisting}
However, the \texttt{NO-CARRIER} flag is also set.
This flag indicates a fault on the physical layer, i.e. a disconnected network cable or similar.
Running a scan on the device shows that it is able to correctly detect nearby networks:
\begin{lstlisting}[language=bash]
>> iw dev wlp2s0b1 scan | grep SSID
SSID: SDU-WEBLOGIN
SSID: eduroam
SSID: SDU-GUEST
\end{lstlisting}
This was believed to be a software related issue, as it could detect networks, but not correctly connect to them.
In an attempt to isolate the error the dongle was connected to a PC.
The dongle worked flawlessly on the PC and connected to the internet after bringing down the standard interface.
Clearly, there are differences between the two systems, one or more of which were causing one to work while the other did not.
In order to eliminate these differences the versions of the involved software was brought up to date.
The PC is running Arch Linux, a distribution using rolling release.
This means that, generally, the software will always be at the latest stable release.
The Xillinux distribution running on the Zybo board is based on the Ubuntu 12.04 release, the long term support version from 2012, see \cite{xillinux}.
As such, much of the software is outdated.
The \texttt{iproute2} package as well as the \texttt{network-manager} were updated to the latest version.
Other than provide some experience with building and searching for dependencies, it did little to alleviate the issues.
\\~\\
It was decided to continue attempting to set up the ad-hoc network on the PC, expecting that this process may provide some insight into the issues on the Zybo board, and how to fix them.
Initially the drivers were verified using the same procedure as mentioned previously, for brevity, this is left out.
From experience, the first step was to disable the network manager.
A network manager is particularly useful when a user wants to connect to the internet or similar, basic operations.
It automatically detects hardware and manages connections in the area.
The automatic setup means that it will often overwrite any changes made by the user.
For this reason the network manager is disabled:
\begin{lstlisting}[language=bash]
>> systemctl connman.service stop
\end{lstlisting}
\texttt{connman} is the network manager used on the system.
In order to create an ad-hoc network it is necessary to change the mode of the interface to IBSS (Independent Basic Service Set):
\begin{lstlisting}[language=bash]
>> ip link set wlp2s0b1 down
>> iw wlp2s0b1 set type ibss
\end{lstlisting}
Notice that the interface naming on this system is following the new "predictable network interface names"-scheme.
This scheme names the interfaces based on various factors such as MAC-address of the hardware; for more details see \cite{interfacenaming}.
The interface needs an address assigned to it in order to communicate on the network.
\begin{lstlisting}[language=bash]
>> ip addr add 192.168.10.9 dev wlp2s0b1
>> ip link set wlp2s0b1 up
\end{lstlisting}
Finally, after having brought up the device, the ad-hoc network can be created:
\begin{lstlisting}[language=bash]
>> iw wlp2s0b1 ibss join go-kart 2412
\end{lstlisting}
This commands creates an ibss (ad-hoc) network called go-kart on frequency 2.412 \si{\giga\hertz}.
Repeating these steps on a different system, assigning a different address to it, makes it possible to ping between the systems: 
\begin{lstlisting}[language=bash]
>> ping 192.168.10.11
PING 192.168.10.11 (192.168.10.11) 56(84) bytes of data.
64 bytes from 192.168.10.11: icmp_seq=1 ttl=64 time=0.592 ms
64 bytes from 192.168.10.11: icmp_seq=2 ttl=64 time=0.638 ms
\end{lstlisting}
Having this up and running, it was attempted to run the above sequence of commands on the Zybo board.
This allowed the Zybo board to correctly create the network.
Trying to ping the board on its assigned IP-adress, however, yielded a \texttt{no route to host} message.
Looking into the routing tables reveals a discrepancy:
\begin{lstlisting}[language=bash]
>> ip route show
default via 192.168.10.1 dev eth0 metric 100 
169.254.0.0/16 dev eth0 scope link metric 1000 
192.168.10.0/24 dev eth0 proto kernel scope link 
	src 192.168.10.10 
192.168.10.0/24 dev wlan0 proto kernel scope link 
	src 192.168.10.10
\end{lstlisting}
\thomas{Get the correct table, wifi is missing}
As can be seen, messages on eth0 and wlan0 are both routed to the same address space, 192.168.10.0/24.
Curiously, this did not pose an issue on the PC's.
Changing the address space to something different corrected the problem.
In this case 196.178.10.0/24 was chosen.
Finally, a short bash script is written to quickly set up the network:
\begin{lstlisting}[language=bash]
>> service network-manager stop
>> ip link set wlan0 down
>> iw wlan0 set type ibss
>> ip link set wlan0 up
>> iw wlan0 ibss join go-kart 2412
>> ip addr add wlan0 196.178.10.10
\end{lstlisting}
Since the system will not necessarily have an interface when it is mounted on the go-kart, it will have to automatically run this script on startup.
On Xillinux, this can be achived by placing the script in \path{/etc/init.d/<script_name>}.

\subsection{Making the connection}~\\
Having correctly set up the network, the next step is to make a connection between the PC and the Zybo.
This can be done using a variety of methods.
For this project, three approaches were considered:

\begin{itemize}
	\item C
	\item Boost
	\item Socat
\end{itemize}

\thomas{What do we actually call this "connection", socket connection, TCP/IP connection, w/e connection?}

\paragraph*{C}has already been used in other parts of the project.
For this reason, this was the first option to be considered.
This option has the benefit of relying only on the standard C libraries.
Although \cite{beej} was used as a reference in setting up the connection, it proved difficult to create a bi-directional connection across the ad-hoc network.

\paragraph*{Boost}is a widely used C++ library.
As the authors had prior knowledge with this library it became the next choice.
ASIO is part of Boost and contains all of the tools necessary to set up a TCP/IP connection.
Boost provides a multitude of different examples and tutorials on how to use ASIO, specifically the chat server from \cite{boostchat} was modified and formed the basis for the program used in setting up the connection\footnote{The code can be found on the associated github repository at \path{/code/socket/} along with instructions on compiling.}.
\thomas{Put server/client code on github:/code/socket/}
This approach was not without problems, however.
As mentioned previously, Xillinux is based on an older version of ubuntu, as such, the version of cmake and Boost available on the platform is insufficient.
In updating the software, one significant issue emerged; the size of the library.
Boost takes up 684MB of space, making it infeasible for use on the ZB.
Clearly, using a library the size of Boost for that one purpose, is unreasonable.

\paragraph*{\underline{So}cket\underline{cat}} is a command line utility that can be used to establish a bi-directional connection between two hosts.
As opposed to the two prior solutions, socat is implemented and verified by the authors of the utility.
In addition, this utility provides exactly the utility needed, as opposed to boost, which is a collection of libraries spanning everything from mathematical algorithms, complex data structures, sockets and numerous other fields.\\
Generally, socat can be called using
\begin{lstlisting}[language=bash]
>> socat [options] <address> <address>
\end{lstlisting}
An address is specified by a keyword and can be of a range of different types.
Of interest in this project is \texttt{TCP} and \texttt{STDIO}.
\texttt{STDIO} can also be represented simply by \texttt{'-'}.
The \texttt{TCP} address requires two arguments, the ip address and the port.
A simple message passing service can be set up in two terminals by first setting up a tcp listener in terminal one:

\begin{lstlisting}[language=bash]
>> socat - tcp-listen:2048
\end{lstlisting}

This command will redirect all incoming data from stdio to port 2048 and conversely, all incoming data from port 2048 is redirected to stdio. 
Connecting to port 2048 on localhost can be done from terminal two:

\begin{lstlisting}[language=bash]
>> socat - tcp:localhost:2048
\end{lstlisting}

The message written in terminal one will appear in terminal two and vice versa.
Using this functionality, the wifi node can be connected to a PC using pipes.
Piping is commonly used on Linux systems to connect the io of programs.
The linux philosophy is that one program should do one thing and do it well.
If more functionality is needed, multiple single-purpose programs are interconnected using pipes.

\subsection{Conclusion}

Throughout this section a connection was made between a PC and the ZB.
It was chosen to use a usb/wifi-dongle to carry the signal on the ZB.
The TP-LINK TL-WN722N, a general consumer dongle, was chosen for two reasons: The chipset of the dongle is supported by the available drivers and it is capable of creating ad-hoc networks.
Using the \texttt{iproute2} package, an ad-hoc network was set up from the the ZB and connected to from a PC.
A small script was created to automatically set up the ad-hoc network upon booting the system.
Three different approaches were attempted in actually creating a connection across the wifi network, C, Boost and finally socat.
Clearly, analysis should have been done before deciding on a method prior to starting the implementation of the socket connection.
C was discarded due to complexity and the time constraints of the project.
While Boost ASIO was made to function on the PC, the size of the library made it infeasible for use on an embedded platform.
Finally, it was decided to use the linux command line utility socat for creating the connection between the PC and the ZB.