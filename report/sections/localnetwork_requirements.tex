%!TEX root = ../main.tex
\todo{Each sentence should be on a new line. That way git more easily merges documents.}
\section{On-vehicle Local Network}

\subsection{Topology}

There are various network topologies that can be used to setup the required node network for this project. These include the ring, bus, mesh ,star and tree topology. Before we specify which one is suitable, a brief description of the purpose and functionality of our network as well as an overview of their advantages and disadvantages are needed. 

\subsubsection{Purpose of the Network}
\todo{Add the abbreviation ov = on-vehicle}
The purpose of this network is to accommodate multiple nodes, such as sensors sub-networks and in general data-producers. These need to be able to communicate with the main on-vehicle computer as well as among them. The reasons for this is that the ov-computer is responsible for transmitting and receiving data from the stationary computer and other nodes may require data from other nodes to cooperate for a task. Also, the ov-computer, apart from communicating, will also handle processing that the rest of the nodes are not able to. 
\\\\
The communication between the various nodes does not require a central hub. Two nodes may need to communicate with one another without the control and the extra processing time of a centralized topology. Furthermore, in the case that one fails, the network as a whole should still be operative. Lastly, since it is a multi-node network and it may require more nodes in the future, a certain level of scalability is also required.

\subsubsection{Different Topologies}
\begin{itemize}
\item \textbf{Bus:} This is the simplest, cheaper and easiest topology. All nodes communicate through a central bus and it is scalable with the addition of extra nodes on the two ends of the cable. The central bus introduces the risk of complete network failure in case of the bus failure and the decrease in performance in case of many nodes or heavy traffic.
\item \textbf{Ring:} All the nodes in this type of network form a ring, where each node is connected to its two neighbors. It has the advantages of the bus network as well and it shows better performance, even under heavy load. A centralized control is not required for this type and in case of a node failure, the ring breaks and it can continue to function as a bus network. It is scalable, but to a less extent than the bus. An alternative solution is the Dual Ring Topology, where a secondary cable is connected in parallel to the main cable, acting as backup.
\item \textbf{Mesh:} In this type, each node has direct connections to each other node present in the network. That provides speed and reliability in case of node failures, but requires more hardware and processing power to manage the connections. It is very robust but scalability is certainly not one of its features.
\item \textbf{Star:} The star topology is a centralized type of networking. All nodes are connected to a central hub, handling all the communication between them. It is fast, easy to implement and offers great reliability in case of node failures. The major disadvantage is that if the hub fails, the entire network fails. It can be scalable to a certain point, since the hub can be upgraded to handle more connections.
\item \textbf{Tree:} The last type, the tree is an extension of the bus and the star topologies combined. It can be easily scalable, but that adds extra difficulty in its maintenance. It also requires a lot of connections and in case the top (root) node fails, then all the network is down as well.
\item \textbf{Hybrid:} The last type is the hybrid one, where depending on the needs and purpose of the network, two or more topologies can be combined to achieve the best balance between their advantages and disadvantages.
\end{itemize}

\subsubsection{Suitable Topology}
For our networking purpose, the mesh type is not suitable, since it adds extra hardware requirements such as processing power and many redundant connections. A node may be a simple sensor with a small microprocessor and hence, connecting it to such a network is not feasible. In our system, a small embedded board computer will be the ov-computer which requires to maintain its connection with the network even in case of failures and also possibly the central hub in centralized topologies that such as the star and the tree. Thus, these two topologies are not suited, since in case of its failure, the whole network fails. Scalability is also a requirement for the future connection of nodes. Although the majority of the types provide a level of scalability, the addition of extra nodes always decrease the overall performance of any network. Hence, a topology that balances the decrease in performance against the network's expansion is best suited for the project.\\\\
The approach that fits the requirements is implementing a bus network, where each node may be a subnetwork of a different type depending on the needs, making it into a hybrid network having a bus topology as its basis. This type provides a good balance of reliability, scalability, hardware requirements and communication speed in comparison to the others.

\subsection{Networking technology}

\subsubsection{Existing Technologies}
Different networking technologies exist in use today, such as Ethernet, CANbus, CANopen and Powerlink, among others. Canbus is widely used in the automotive industry with data rates up to 1Mbit/s for small networks lengths, but the classic Ethernet and Powerlink support data rates of at least 10Mbit/s. Furthermore, Powerlink is suitable for transmission of time-critical data and timing-synchronization of the nodes and CANopen networks are very robust and reliable.
\subsubsection{Choosing a Technology}
The two choices that were considered for the scope of this project were Powerlink and CANopen. The first one can be implemented using openPowerlink at a software level and utilizing PmodNIC100 modules provided by the university. On the other hand, CANopen can be implemented using CANopenSocket\footnote{https://github.com/CANopenNode/CANopenSocket}, an open source git project built on CANopenNode. This library is easily implemented in Linux and can use various hardware, one of them being a USB to CAN interface board named USBtin\footnote{http://www.fischl.de/usbtin/}. Another available module to use is the CANbus tranceiver breakout board from Copperhill which can be used by an IP core at a VHDL level, greatly increasing the performance of such a network. Since CANbus is a technology widely used in the automotive industry, it is our first choice for developing our system.


%Although for the purpose of this project Ethernet suffices, Powerlink adds a level for critical timing synchronization, a feature certainly needed for nodes that need to communicate and to react fast to a situation such as braking or to log data on a very small and specific time interval. Since the Zybo board will be used possibly using PmodNIC100 modules as they are provided by the university, the openPowerlink protocol can be implemented in software utilizing as hardware the existing Ethernet ports provided by the modules. CAN-bus is also an option to implement with for example the CAN-bus tranceiver breakout board from Copperhill, and since it is a technology widely used in the automotive industry.
%link to CAN tranceiver: http://copperhilltech.com/can-bus-mini-breakout-board/
%\todo[inline]{Martin: I'm not so sure, at the ring network wins over the bus on any of those points. Both the bus and ring can only have one node sending at a time, the whole thing goes to shit if one cable is cut, the ring actually requires two ethernet ports, whereas the bus only requires one. I would agree though, that the speed is significantly better, but that's thanks to eternet.}


