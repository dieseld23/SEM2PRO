%!TEX root = ../main.tex

\section{Data Rate}\label{sec:data_rate}
This section will explore the amount of data that can be expected from a data producer connected to the system.
Many of these parameters of interest pose different requirements in terms of the desired sample rate as well as the size of each sample.
A temperature sensor, for instance, does not require nearly the same sampling frequency as an accelerometer.
Due to these differences the data rate expected from one data producer may differ wildly from another.
In order to provide a safe estimate, the expected data rate is calculated using a worst case sensor.
That is, the type of sensor which sets the highest requirements in terms of sampling frequency as well as data size.
Again, it is not possible to foresee every application that may be developed in the future.
Despite of this the IMU chosen for this project, see section~\ref{sec:imu}, is considered as the worst-case.
It is is a 9-axis sensor that provides data at a rate up to 300 Hz at 32 bit floating point precision.
This would result in a data rate of:

$$300\cdot32\cdot9=86.4\,\text{Kb/s}$$

As previously mentioned, this is the assumed worst case for the system.
As such there may be only a few data producers providing data at this rate.
Most other producers will provide only limited data in relation to the IMU, either due to a much lower sampling frequency or an overall smaller data size.
A GPS generally provides an update only at a few \si{\hertz}. 
An encoder, while it may have a reasonably high sampling rate, it is most likely not close to 32 bit resolution.\\

Assuming five sensors running at the assumed worst case rate and ten running at half that rate, the resulting data rate will be 864 kb/s.