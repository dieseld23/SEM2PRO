%!TEX root = ../main.tex

\section{Embedded Platform}\label{sec:EP}
As described in section \ref{sec:data_collection}, a CAN network will be created to support the nodes on the go-kart.
Each node is some type of embedded platform, possibly connected to a sensor or other data producing unit.
From the perspective of the network the only requirement for the embedded platform is that it has a CAN controller.
In principle the network could consist of 16 different embedded platform's.
\subsection{Nodes}
In section \ref{sec:hardware_for_par}, it was chosen to use an IMU and a GPS, both using a USB interface.
Additionally, one node has to transfer data wirelessly from the go-kart to a stationary monitoring system.
In section~\ref{sec:wifi}, WiFi is chosen as the wireless protocol, and this will be handled on a separate node.
Both USB and WiFi are implemented in Linux.
Thus choosing an embedded platform supporting such an operating system will greatly simplify the code that needs to be written for all nodes.
The authors attend a course on embedded software design in Linux, as such, this is a natural choice.\\

As a part of the aforementioned course, the authors have access to Zybos.
This platform fits the requirements set for a node and being accessible, this is the EP chosen for the project.

\subsection{Zynq Board (Zybo)}
The Zybo has a Xilinx Zynq Z-7010 chip, which has a processing system (PS) part and a programmable logic (PL) part.
The PS consists of a dual-core ARM Cortex-A9 processor and I/O peripherals, including two CAN controllers and USB.
The PL consists of a Xilinx 7-series Field Programmable Gate Array (FPGA). 
The PL and PS are connected through the AXI bus.
The Zybo itself is made as a development platform with several buttons, switches, LEDs, connections for USB, Ethernet, HDMI and several PMOD connectors.
\subsubsection*{Linux on the Zybo}
Several Linux distribution are configured to run on the Zybo. 
The authors have experience with the Xillinux\cite{xillinux} distribution, therefore this will be used.
Xillinux is based upon the digilent Linux distribution which is built on the Xilinx distribution which is based upon Ubuntu 12.04.
Xillinux architecture implements the Xillybus which is a DMA-based bus between the PL and PS.
Xillybus is implemented as an IP core in PL and a corresponding Xillybus driver in the Xillinux kernel.
In PL the IP core is interfaced through standard FIFO buffers.
In PS the Xillybus is reached in userspace through \texttt{/dev/xillybus\_<bus-name>}.



\subsubsection*{CAN Controller on the Zybo }
\mikkel{Moved! Changede by Catalin and Mikkel}
The Zybo provides two different types of CAN Controllers.
Xilinx provides an AXI Core implementing a CAN controller in PL and it has a built-in CAN Controller in the PS.
The AXI core CAN Controller is available in the Vivado Suite, but it cannot actually be bitstreamed without obtaining the appropriate license from Xilinx.
The CAN Controller in PS is readily usable and will therefore be used in this project.

\subsection{Power Requirements}\label{sec:power_requirements}
The Zybos need to be supplied with 5 V, which subsequently powers the sensors, the CAN bus and any other peripherals present on the go-kart.
The Sevcon has a 5V output capable of supplying up to $\si{100 \milli \ampere}$, which is insufficient.
It is therefore necessary to use a DC-DC converter to supply the nodes.
Some tests along with datasheet lookups are used to determine the power requirements for the system.\\

By measurement, it was found that one Zybo draws a maximum of $\si{475 \milli \ampere}$ while running Linux.
The CAN transceivers consume up to 40 mA\cite{3.3V_CAN}. 
The GPS module consumes up to 40 mA, the IMU consumes up to 56 mA. 
Generally it is assumed that the sensors do not consume more than 75 mA per node, which brings the total current consumption up to 600 mA.
For 16 nodes, this brings the total current up to 9.6 A.\\

The source for this DC-DC converter will be the batteries on the go-kart. 
These vary depending on charge from 57.6 down to 40 V, so the DC-DC converter would need to be operable in this entire range.


\begin{table}[H]
	\centering
	\begin{tabular}{r|l}
		Parameter & Requirement  \\
		\hline
		Input voltage & 40-57.6 V\\
		Output voltage & 5 V\\
		Output Current & 9.6 A
	\end{tabular}
\end{table}


\subsection{Conclusion}
A node is not hardware specific.
It can be run on any type of embedded platform, so long as it is capable of communication over CAN.
Throughout this project the nodes will be run on the Zybo platform.
This will ease the development as Linux includes support for both USB and WiFi.
An analysis of the power draw of a fully populated network revealed that a supply of upwards of 10\si{\ampere} is required.
