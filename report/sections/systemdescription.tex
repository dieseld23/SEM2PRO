%!TEX root = ../main.tex

\section{System description}
%\subsection{Communication of sensor data and parameters between a computer on the SDU go-kart and a stationary computer.}
%Bla Bla....
\subsection{What is the purpose of transferring go-kart data to and from a stationary computer?}
To let an engineer or a mechanic monitor the status of the go-kart.
Movement data could be usefull for testing new hardware, improving existing hardware and evaluation the drivers performance. 
Data about battery status, currents, motortemperature is usefull for safety procedures.
Data from the SEVCON xxx inverter on the go-kart could be used for tuning parameters, evaluation performance and safety measures.
On the basis of this data it should be possibly to change configuration parameters on the SEVCON xxx while driving the go-kart.

\subsubsection{What data and sensors would be interesting to monitor?}
Data from IMU, GPS and encoders would be necessary to monitor the movement of the go-kart.
Data from the SEVCON xxx gives information about about battery status, currents, motortemperature.

\subsubsection{What kind is the stationary computer?}
It should be possible for different personnel to monitor the go-kart data therefore the statinonary computer should just be general purpose computer running (linux?).

\subsubsection{What kind of communcation is needed to transfer data between the go-kart computer and the stationary computer?}
As it should be possibly to monitor data and change parameters while the go-kart is driving the connection clearly needs to be wireless with a range allowing for driving on a typical racing track. 
As all data is used by humans the propagation delay of the connection is not required to be "superfast" (should be rewritten in numbers somewhere).

\subsubsection{How should data be collected from multiple sensors?}
???? I need a little bit help with this?
Something with local data collection as this data could be used for other things such as localization and local safety procudures.
There coulde be somthing abotu data collection here.
What is the role of the local computer and why?


%\subsubsection{What is the data on stationary computer intended for?}
%Monitoring data related to movement from the go-kart and adjusting go-kart control parameters by an engineer.

%\subsubsection{What kind of data}
%Data from Sevcon controller related to the motor (control, temperature, speed etc.), and from other nodes that are sensors. Sensors may include IMU, GPS... (to be discussed).

%\subsubsection{What is meant by "computer"}
%Stationary computer is a laptop/desktop (general purpose computer, OS: Windows/Linux), and the go-kart computer is a zybo board running Linux.

%\subsubsection{What setup should be used for communication (Wireless/wired), what protocol etc.}
%Wireless communication to make possible the monitoring/adjusting while the go-kart is being driven on a track. Protocol?

%\subsubsection{Does the communication require extra verification? checksum, timestamping etc.?}
%YES.

\subsubsection{Is storage necessary?}
To be discussed.

\subsubsection{How many nodes/sensors should the system be able to handle?}
The system will be scalable. How much will be discussed.

%\subsubsection{What kind of nodes/sensors/actuators?}
%Sevcon and possibly IMU, GPS...

\subsubsection{QUESTIONS}
\begin{itemize}{}
\item What bandwidth/latency should the communication be able to handle?
\item How is data produced? i.e. just by the sensors, by the go-kart or both systems?
\item How many sensors/data producers?\\
\item Should we support asynchronous transfer? (different sensors with different 
update frequency transfer at different rates).\\
\item Which tasks should be handled where? i.e. can some tasks be handled locally on 
the go-kart?
\item What topology should the go-kart network be? Ring?
\item What should the basic gui (on the laptop) be?
\item Go-kart network hierarchy? A zybo master ruling the rest puny zybos without mercy?
\item Required hardware to make the entire network? (eg. Zybos, pmod ip core ethernet, wifi card for zybo etc.)
\item Sevcon requires use of OpenCan. How will the connection between the Sevcon and the network should be done?
\item What is the max distance we want to support for wireless communication Stat. Laptop - Go-kart zybo? What hardware for that distance?
\item How does this max distance affects the bandwidth/latency and in general, the efficiency of the entire network?
\item If we have error-handling on the go-kart zybo, what kind of errors might those be?!??!
\item Programming language/IDE of creating the GUI?
\item How the failure of a node will be handled?
\item How the wifi disconnection between Stationary PC and the go-kart zybo will be handled? A msg in GUI? What about if data was being transmitted, should we check if the node got the data? In other words, do we need verification of data received by the nodes shown on the gui after adjusting the parameters?

\end{itemize}


\section{Further ivestigation}
