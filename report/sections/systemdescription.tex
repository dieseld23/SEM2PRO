%!TEX root = ../main.tex

\section{Communication of sensor data and parameters between a computer on the SDU go-kart and a stationary computer.}
This project aims to produce a tool for developing, evaluation and testing the SDU go-kart.
Movement of the go-kart should be monitored to enable evaluation of the performance of the driver.
The control variables and currents of the inverter should be monitored in order to help develop new hardware or evaluate existing hardware.
Relavant control variables should be controllable, while driving allowing for efficient testing controllers.
To achieve this data should be collected on a computer on the go-kart and sent to a stationary computer.

%\subsection{What is the role of transferring go-kart data to and from a stationary computer?}
%To let an engineer or a mechanic monitor the status of the go-kart.

%Movement data could be usefull for testing new hardware, improving existing hardware and evaluation the drivers performance. 
%Data about battery status, currents, motortemperature is usefull for safety procedures.
%Data from the SEVCON xxx inverter or other inverters on the go-kart could be used for tuning parameters, evaluation performance and safety measures.
%On the basis of this data it should be possibly to change configuration parameters on the SEVCON xxx while driving the go-kart.

\subsection{What is the role of the stationary computer?}
The stationary computer should be responsible of the transmission of relevant go-kart data to and from the "on-board" computer.
The received data should be presented to the user and the user should be able to change relevant contral variables.

\subsubsection*{What data and sensors would be interesting to monitor?}
Data from IMU, GPS and encoders would be necessary to monitor the movement of the go-kart.
Data from the SEVCON xxx gives information about about battery status, currents, motortemperature.
%Temperature of tyres, speed of front wheels, possibility to control or at least moniter stearing and brake.

\subsubsection*{What kind is the stationary computer?}
It should be possible for different personnel to monitor the go-kart data therefore the statinonary computer should just be general purpose computer running (linux?).


\subsubsection*{What control variables would be interesting to change?}
......

\subsubsection*{What kind of communication is needed to transfer data between the go-kart computer and the stationary computer?}
As it should be possibly to monitor data and change parameters while the go-kart is driving the connection clearly needs to be wireless with a range allowing for driving on a typical racing track. 
As all data is used by humans the propagation delay of the connection is not required to be "superfast" (should be rewritten in numbers somewhere).

\subsubsection*{How should data on the stationary computer be presented?}
As live data should be easy to read? ("overskue" in danish) there needs to be a GUI or at least a UI that will present the user with easy readable data. 
Changing of parameters on the go-kart should also be in the UI in order useable by others than the developers.

\subsection{What is the role? of the computer on the go-kart?}
The "onboard" computer needs to handle both the collection
of data from different sensors (data producers) and the transmission of data to 
data consumers.
Data consumers may be sensors that require setup, filters or similar equipment.
This computer will also handle the transmission of data to the stationary 
computer.
%There are tasks that need to be handled locally fx strict safety measures (and future things fx localisation) that requires a miminum delay. 
Storing of data with a 
The computer needs also to store data locally in order to avoid data loss if the wireless connection is lost.

\subsubsection*{How is local data transmission handled?}
All data should be collected locally on the "on-board" computer to transmit it to the stationary computer.
Local data collection should be "fast" and "reliable" for local safety measures to function properly.
It should be possibly to collect data from and transmit data to at least the 
aforementioned IMU, GPS and SEVCON xx with the possibility of adding additional 
data producers/consumers.

\subsubsection*{What kind is the "on-board" computer?}
The computer should be a small embedded platform in order to mount it physcically on the go-kart.
It should be able to handle all "on-board" tasks.

\subsubsection*{How should local data storage be realised?}
Data should be stored directly on nonvolatile memory to be able retrieve data after shutdown.
Data integrity checks is needed to ensure proper data logging.
%Correct data logging should be realised to ensure validity of data. 

\newpage
\section{Further ivestigation}

\subsection{Parameters of Interest}
In developing new hardware or evaluating current hardware, it is necessary to be able to monitor a range of parameters.
This section will investigate what parameters need to be logged in order to provide a useful and complete logging of the behaviour of the go-kart.
The parameters in question fall into three categories; Physical parameters, electrical parameters and mechanical parameters.
These will be dealt with in turn in the following sections
\paragraph*{Physical Parameters}
This category comprises all information about the motion of the go-kart.
\begin{itemize}
	\item \textbf{Position, Absolute:} Providing a means to record the absolute position of the go-kart is a useful feature in certain fields.
	Especially any form of localisation and pathfinding will be able to put this information to use.
	The absolute position of the go-kart can be recorded using a GPS module or possibly by using a known starting coordinate and information about the relative movement of the go-kart.
	\item \textbf{Position, Relative:} The relative position of the kart can be, as just mentioned, used to infer the absolute position of the go-kart.
	Additionally it can provide a means to analyze a drivers performance on the track or detect drift while cornoring.
	The relative position includes both translational, as well as rotational information.
	This information can be gathered using an inertial measurement unit (IMU).
	An IMU is a compound device, comprising of an accelerometer and a gyroscope and, in some cases, a magnetometer.
	\item \textbf{Velocity:} The velocity of the go-kart is key in optimising lap-times, clearly, it is desirable to monitor this parameter.
	It can be extracted by reading the motor encoders.
	However, the driving wheels are prone to slippage when cornoring, this would give an inaccurate reading of the actual velocity of the go-kart.
	Instead, a simple encoder can be mounted on either one, or both of the front wheels as these are freerunning and independent.
	Once the rotational speed of the axle is known, the velocity of the go-kart can be infered using the tyre diameter.
	\item \textbf{Acceleration:} It may be of interest to monitor the forces exerted on the go-kart, or, its acceleration, as it drives on the track.
	This information is already provided by the accelerometer in the IMU mentioned above and as such provides no additional complication.
\end{itemize}
\paragraph*{Electrical Parameters}
This category comprises all information about the electrical aspects of the go-kart.
\begin{itemize}
	\item \textbf{Motor Currents:} Providing a means of monitoring the currents flowing through the motor allows the user to calculate the torque exerted by the motor as well as the current power draw of the motor.
	Knowing the currents could also prove an invaluable debugging tool when developing a new inverter for the go-kart.
	These currents are all monitored by the sevcon gen4 controller mounted on the go-kart.
	As such the information is readily available through the CAN interface of the controller.
	\item \textbf{Throttle Position:} The throttle on the go-kart is connected to a potentiometer.
	Measuring the voltage output of this potentiometer provides a simple way of monitoring the position of the throttle.
	\item \textbf{Desired Currents:} Based on the current throttle position a set of desired currents are calculated.
	Monitoring these allows spotting any discrepancies between the desired and the actual currents.
	\item \textbf{Battery Voltage:} As the go-kart is electrical, naturally, it has a battery.
	Monitoring the current battery status could give the user an indication of how much driving time is left, or how long until the batteries are recharged afterwards.
	\item \textbf{Motor Angle:}
\end{itemize}
\paragraph*{Mechanical Information}
This category comprises all information about the mechanical aspects of the go-kart.
\begin{itemize}
	\item \textbf{Steering Wheel Angle:} Monitoring the angle of the steering wheel allows analysing the performance of the driver.
	In addition it opens up for the possibility of mechanical control of the go-kart.
	Similarly to monitoring the velocity, the steering wheel angle can be monitored by adding an encoder to the steering column.
	\item \textbf{Braking Pedal Position:} The braking system on the go-kart is similar to that of an ordinary car.
	The braking disc is mounted on the driving axle and the braking calibers connected to the brake pedal by a series of oil-filled hoses.
	Monitoring its actuation allows analysing the performance of the driver and as mentioned above, may potentially allow for mechanical control of the go-kart
\end{itemize}

%\subsubsection{What is the data on stationary computer intended for?}
%Monitoring data related to movement from the go-kart and adjusting go-kart control parameters by an engineer.

%\subsubsection{What kind of data}
%Data from Sevcon controller related to the motor (control, temperature, speed etc.), and from other nodes that are sensors. Sensors may include IMU, GPS... (to be discussed).

%\subsubsection{What is meant by "computer"}
%Stationary computer is a laptop/desktop (general purpose computer, OS: Windows/Linux), and the go-kart computer is a zybo board running Linux.

%\subsubsection{What setup should be used for communication (Wireless/wired), what protocol etc.}
%Wireless communication to make possible the monitoring/adjusting while the go-kart is being driven on a track. Protocol?

%\subsubsection{Does the communication require extra verification? checksum, timestamping etc.?}
%YES.


%\subsubsection{How many nodes/sensors should the system be able to handle?}
%The system will be scalable. How much will be discussed.

%\subsubsection{What kind of nodes/sensors/actuators?}
%Sevcon and possibly IMU, GPS...


\subsection{What will the data be used forr (should maybe be above some other questions?)}



