%!TEX root = ../main.tex
\section{Adding a New Node to the System}
\label{app:addnode}
In adding a new node to the system there are two steps that need to done:
\begin{itemize}
	\item Prepare the sensor node software.
	\item Add appropriate functions to the API.
\end{itemize}
\mikkel{Update OD also?}
This appendix will explain those steps to enable a user to add their custom nodes to the network.

\subsection{Preparing sensor node software}\label{sec:node_implementation_guide}
The general node software is described in section \ref{sec:sensor_node} and the class diagram in figure \ref{fig:node_class_diagram}.
When developing a new node with another sensor the generic classes CAN\_link, Protocol and Node should be used. 
New classes should be developed to extract data from sensor, pack data according to protocol and put the data in a data\_packet.
It is advantageous to implement this functionality into a Packer\_Sensor and a Sensor class to keep a similar software design through the nodes.
The interface to the generic Node class is a call to its function \texttt{put\_data\_packet(data\_packet)}.

\subsubsection*{Sensor class}
A Sensor class should be implemented to interface the sensor. 
This class will be very sensor specific and not generic. 
It should update its own variable with all relevant from the sensor.
This functionality should be implemented in a loop function that runs in a thread and calls a function in the Sensor\_Packer class when data is updated.


\subsubsection*{Sensor\_Packer class}
The Sensor\_Packer class should also be implemented.
It is sensor specific, because it needs to pack sensor data according to the specific sensors messagetypes.
A Sensor\_Packer class should inherit from the base Packer class shown in code snippet \ref{code:packer}.
In the constructor of the Sensor\_Packer all messagetypes for the sensor should be put in \texttt{messagetypes}.

\begin{lstlisting}[caption=Packer class.,label=code:packer]
class Packer {
protected: 
	std::vector<std::bitset<6> > messagetypes;
public:
	Node* node;
	void set_node(Node* node_in);
};
\end{lstlisting}

The class needs to pack data into the \texttt{data\_packet} shown in code snippet \ref{code:data_packet}.
It is responsible to populate the variables \texttt{messagetype} and \texttt{data}.
To do that all data that is to be sent needs to be converted into \texttt{vector<bool>}.
When \texttt{data\_packet} is populated correctly it needs to be passed to the generic Node class using its member function \texttt{put\_data\_packet(data\_packet)}.

\subsection{Creating the API}
The front end is comprised of a list of functions which enable the user to access the data collected on the network.
The data, up until the front end is not in human readable format.
Each function in the API reads the appropriate data and converts it to a more useful format.
A template of an API function is given in code \ref{code:apitemplate}.

\begin{lstlisting}[caption=Function template for accesing data,label=code:apitemplate]
#define <MESSAGE_ID> "11110001001"

<PARAMETER_TYPE> backend::get_<PARAMETER>()
{
	std::vector<data_t> v;

	while(!fifo_o->get(<MESSAGE_ID>, v));

	if (!v.empty())
	{
		double t = std::stoi(v.back().millis);
		std::string d = v.back().data;

		/*Decipher d*/		
	}

	return <PARAMETER_TYPE>;
}
\end{lstlisting}

The definition of the message id of the message being decoded should be placed in \texttt{backend.hpp}.
The return type of functions is optional.
The \texttt{get\_coordinate()} function implemented, for example, returns a struct containing the relevant data.
All data is kept in a vector with 0-1024 elements where the last element of the vector is always the newest.
At line 5 a container for this vector, \texttt{v} is instantiated.
The while loop will continue to run until \texttt{v} is populated.
This will not happen until the \texttt{get} function succesfully locks the mutex.
Each entry in \texttt{v} is a struct as seen in code \ref{code:data_t}

\begin{lstlisting}[caption=Struct used to store each datapoint,label=code:data_t]
struct data_t
{
	std::string millis;
	std::string data;
};
\end{lstlisting}

In this struct, \texttt{millis} is the number of milliseconds since startup in binary.
\texttt{data} is the binary form of the associated data given by \texttt{MESSAGE\_ID}.
Once \texttt{v} is populated the data can be accessed as shown in lines 11 and 12.