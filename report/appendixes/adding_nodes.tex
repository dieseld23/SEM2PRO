%!TEX root = ../main.tex
\section{Adding a New Node to the System}
\label{app:addnode}
In adding a new node to the system there are two steps that need to done:
\begin{itemize}
	\item Prepare the node software.
	\item Add appropriate functions to the API.
\end{itemize}
This appendix will explain those steps to enable a user to add their custom nodes to the network.
\subsection{Preparing node software}\label{sec:node_implementation_guide}
\thomas{Here goes you, Mikkel!}

\subsection{Creating the API}
The front end is comprised of a list of functions which enable the user to access the data collected on the network.
The data, up until the front end is not in human readable format.
Each function in the API reads the appropriate data and converts it to a more useful format.
A template of an API function is given in code \ref{code:apitemplate}.

\begin{lstlisting}[caption=Function template for accesing data,label=code:apitemplate]
#define <MESSAGE_ID> "11110001001"

<PARAMETER_TYPE> backend::get_<PARAMETER>()
{
	std::vector<data_t> v;

	while(!fifo_o->get(<MESSAGE_ID>, v));

	if (!v.empty())
	{
		double t = std::stoi(v.back().millis);
		std::string d = v.back().data;

		/*Decipher d*/		
	}

	return <PARAMETER_TYPE>;
}
\end{lstlisting}

The definition of the message id of the message being decoded should be placed in \texttt{backend.hpp}.
The return type of functions is optional.
The \texttt{get\_coordinate()} function implemented, for example, returns a struct containing the relevant data.
All data is kept in a vector with 0-1024 elements where the last element of the vector is always the newest.
At line 5 a container for this vector, \texttt{v} is instantiated.
The while loop will continue to run until \texttt{v} is populated.
This will not happen until the \texttt{get} function succesfully locks the mutex.
Each entry in \texttt{v} is a struct as seen in code \ref{code:data_t}

\begin{lstlisting}[caption=Struct used to store each datapoint,label=code:data_t]
struct data_t
{
	std::string millis;
	std::string data;
};
\end{lstlisting}

In this struct, \texttt{millis} is the number of milliseconds since startup in binary.
\texttt{data} is the binary form of the associated data given by \texttt{MESSAGE\_ID}.
Once \texttt{v} is populated the data can be accessed as shown in lines 11 and 12.
